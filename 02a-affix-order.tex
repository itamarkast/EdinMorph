\documentclass[a4paper,12pt]{article}

%\usepackage{times}
%\usepackage[T3,T1]{fontenc}
%\usepackage[utf8x]{inputenc}
% \usepackage[T1]{fontenc}
% \usepackage{mathptmx}  

\usepackage{fontspec} %,xunicode} % xltxtra
    \defaultfontfeatures{Ligatures=TeX}
    \setmainfont{Brill}[
    Extension=.ttf,
    UprightFont=*-Roman,
    BoldFont=*-Bold,
    ItalicFont=*-Italic,
    BoldItalicFont=*-Bold-Italic,
    Renderer=ICU]
% 	\usefonttheme{serif}
%	\setmainfont[Renderer=ICU]{Charis SIL}
% 	\setmainfont[Renderer=ICU]{Brill}

\usepackage{amsmath,amssymb} % for $\text{}$
\usepackage{url,natbib} 
\usepackage[dvipsnames]{xcolor} %,bm}
\usepackage{geometry,vmargin,setspace}
\usepackage{multirow}
\setmarginsrb{1in}{1in}{1in}{1in}{13.6pt}{0.1in}{0.1in}{0.2in}
% pdflscape} %rotating
    \setlength{\parindent}{0pt}

\usepackage{stmaryrd}
\usepackage{wasysym} %checkbox
\usepackage[normalem]{ulem} % for \sout{}
% \usepackage{mdwlist} % for \begin{itemize*} - but prefer \tightlist
\providecommand{\tightlist}{%
	\setlength{\itemsep}{0pt}\setlength{\parskip}{0pt}}

\usepackage{natbib}
	\bibpunct[:]{(}{)}{;}{a}{}{,}
	\setlength{\bibsep}{0pt plus 0.3ex}

\usepackage{longtable}
\usepackage[linkcolor=purple,citecolor=ForestGreen,colorlinks=true,urlcolor=gray,pagebackref=true]{hyperref}
\usepackage{multicol}
\usepackage{dashrule} %\hdashrule
\usepackage{array} % >{\...} in tabulars
\usepackage{arydshln}

\usepackage[framemethod=tikz,footnoteinside=false]{mdframed} 

\usepackage{expex} %Linguistics examples
\lingset{aboveexskip=0ex,belowexskip=0.5ex,aboveglftskip=-0.3em,interpartskip=0ex,labelwidth=!6pt,belowpreambleskip=0.1ex} %, *=*?}
%	\lingset{aboveexskip=0.5ex,belowexskip=0.5ex,*=??}

\usepackage[nocenter]{qtree} % trees
\usepackage{tikz}
    \tikzstyle{every picture}+=[remember picture]

\usepackage{tipa} % convenient for \textsubarch etc.

\newcommand\trace{\rule[-0.5ex]{0.5cm}{.4pt}}
\newcommand\midline{\rule[-0.5ex]{4cm}{.4pt}}
\newcommand\longline{\rule[-0.5ex]{7cm}{.4pt}}

\bibpunct[:]{(}{)}{;}{a}{}{,}
	\setlength{\bibsep}{0pt plus 0.3ex}

\usepackage{pifont}
\newcommand{\cmark}{\ding{51}}% 52
\newcommand{\xmark}{\ding{55}}%
 \newcommand{\hand}{\ding{43}}

\newcommand\zero{\O{}}
\newcommand\itp[1]{\textit{\textipa{#1}}}
\newcommand\gsc[1]{\textsc{\lowercase{#1}}} %for glossing in small caps - comment out to return to caps
\renewcommand\root[1]{$\sqrt{\text{#1}}$}

\newcommand\blue[1]{\textcolor{blue}{#1}}
\newcommand\red[1]{\textcolor{red}{#1}}
\newcommand\green[1]{\textcolor{ForestGreen}{#1}}
\newcommand\gray[1]{\textcolor{gray}{#1}}
\newcommand\denote[1]{$\llbracket$#1$\rrbracket$}
\newcommand\lra{$\leftrightarrow$}

\newcommand\vz{\text{Voice$_{\text{\{--D\}}}$}}
\newcommand\vd{\text{Voice$_{\text{\{+D\}}}$}}
\newcommand\pz{\text{$p_{\text{\zero}}$}}
\newcommand\va{\root{\gsc{ACTION}}}
\newcommand{\tkal}{\emph{XaYaZ}}
\newcommand{\tpie}{\emph{XiY̯eZ}}
\newcommand{\tpua}{\emph{XuY̯aZ}}
\newcommand{\thif}{\emph{heXYiZ}}
\newcommand{\thuf}{\emph{huXYaZ}}
\newcommand{\thit}{\emph{hitXaY̯eZ}}
\newcommand{\tnif}{\emph{niXYaZ}}
\newcommand\dgs[1]{\textsubarch{#1}}
\newcommand\del[1]{\sout{#1}}

\makeatletter %Allow superscript ^ and subscript _
\catcode`_=\active%
\gdef_#1{\ensuremath{{}\sb{#1}}}%
\catcode`^=\active%
\gdef^#1{\ensuremath{{}\sp{#1}}}%
\makeatother

\begin{document}
% \pagenumbering{gobble}
% \singlespacing


\hfill \emph{Morphology 2022-23, Edinburgh}

\section*{Warm-up}

Segment the following words.

\pex
    \a walk
    \a walks
    \a talk
    \a talks
    \a walking
    \a walkable
    \a walkability
\xe

And what issues arise when we segment these ones?

\pex
    \a bassoon
    \a balloon
    \a buffoon
    \a baboon
\xe
% Based on some of Jonathan Bobaljik's materials, not sure where it originated!

\pex
    \a glow
    \a glisten
    \a glimmer
\xe

Not every string that repeats itself is a grammatical primitive (a morpheme). How can we tell? What are we looking for? What kinds of evidence are relevant?

\section{Affix ordering}

    \subsection{Auxiliaries and preliminaries}
What's the order of auxiliaries in English?

\pex
    \a She will have be-en winn-ing the race.
    \a \ljudge{*} She has been will winning the race.
\xe
\pex
    \a The cake will have be-en (be-ing) eat-en.
    \a \ljudge{*} The cake is having were been eat.
\xe

\midline{}

\bigskip
How each element fits with the others:
% \begin{itemize} \tightlist
%     \item Takes semantic \emph{scope} over the next: the future of the perfect of the passive of an eating event.
%     \item Takes morpho(-phono)logical \emph{scope} over the next: \emph{have}_{\gsc{perf}} triggers participial morphology on \emph{eat-en}, \emph{is}_{\gsc{prog}} triggers progressive morphology on \emph{winn-ing}, etc.
% \end{itemize}

\vspace{3cm}

\bigskip
Representing this formally:

\newpage
\ex
% \Tree
% [.TP
%     [.\emph{The cake} ]
%     [.
%         [.T\\\emph{will} ]
%         [.PerfP/AspP
%             [.Perf\\\emph{have} ]
%             [.ProgP
%                 [.Prog\\\emph{been} ]
%                 [.PassP
%                     [.Pass\\\emph{being} ]
%                     [.VP\\\emph{eaten} ]
%                 ]
%             ]
%         ]
%     ]
% ]
\xe
\vspace{3cm}

What's the order in Latin \citep{embick10,kastnerzu17,kastner18nllt}?
\pex
    \a \begingl
        \gla am-\=a-ve-ra-m//
        \glb \root{LOVE}-\gsc{THEME}-Perf-Past-1\gsc{SG}//
        \glft `I had loved'//
        \endgl
    \a \begingl
        \gla am-\=a-ve-r-\=o//
        \glb \root{LOVE}-\gsc{THEME}-Perf-Fut-1SG//
        \glft `I will have loved'//
        \endgl
\xe

\vspace{3cm}

% V < Th < Perf < T.
% \begin{itemize} \tightlist
%     \item $\approx$ T > Perf > V.
%     \item Like in English, except in morphology rather than syntax.
%     \item Still get morphological conditioning, but we'll return to that later (allomorphy).
% \end{itemize}

How does this compare to the order in English?

\longline{}

    \subsection{The Mirror Principle}
        \subsubsection{Preliminaries}
What's the order of the reciprocal and causative suffixes in Chiche\^{w}a \citep{alsina99}?
\ex[exno=3] \begingl
    \gla Al\=enje a-na-mény-\textbf{án}-\uline{its}-á mb\^{u}zi//
    \glb 2.hunters 2\gsc{s}-\gsc{PAST}-hit-\gsc{RECIP}-\gsc{CAUS}-\gsc{FinVwl} 10.goats//
    \glft `The hunters made the goats hit each other.'//
    \endgl
\xe
\ex[exno=4] \begingl
    \gla Al\=enje a-na-mény-\uline{éts}-\textbf{an}-a mb\^{u}zi//
    \glb 2.hunters \gsc{2S}-\gsc{PAST}-hit-\gsc{CAUS}-\gsc{RECIP}-\gsc{FV} 10.goats//
    \glft `The hunters made each other hit the goats.'//
    \endgl
\xe

\longline{}
% The order is whatever it needs to be to get scope right; You go from the verb ``outwards''.

\bigskip
Here's another pair \citep{hymanmchombo92}:
\pex
    \a mang-\textbf{an}-\uline{its}\\
    tie-\gsc{RECIP}-\gsc{CAUS}\\
    `cause to tie each other'
    \a mang-\uline{its}-\textbf{an}\\
    tie-\gsc{CAUS}-\gsc{RECIP}\\
    `cause each other to tie'
\xe

What would this language (\emph{Chichewa'}) look like if it had prefixes instead of suffixes?
\pex
    \a \midline{} % its-an-mang
    \a \midline{} % an-its-mang
\xe

% Because hierarchy is what matters (relative ordering), rather than linear order.

\bigskip

Another example: Bemba in \cite{baker85}, citing \cite{givon76}.
\pex[exno=49]
    \a \begingl
        \gla Naa-mon-an-ya Mwape na Mutumba//
        \glb 1s.\gsc{S}-\gsc{past}-see-\gsc{recip-caus} Mwape and Mutumba//
        \glft `I made Mwape and Mutumba see each other.'//
        \endgl
    \a \begingl
        \gla Mwape na Chilufya baa-mon-eshy-ana Mutumba//
        \glb Mwape and Chilufya 3p.\gsc{S}-see-\gsc{caus-recip} Mutumba//
        \glft `Mwape and Chilufya made each other see Mutumba.'//
        \endgl
\xe

        \subsubsection{Passivizing}
In Chichewa, the applicative can be used for instruments \citep{alsina99}:
\pex[exno=10] 
    \a \begingl
        \gla Ms\=odzi a-na-dúl-\textbf{ír}-a \textbf{nkhw\^{a}ngwa} uk\=onde//
        \glb 1.fisherman \gsc{1S}-\gsc{PAST}-cut-\gsc{APPL-FV} 9.axe 14.net//
        \glft `The fisherman cut the net with an axe.'//
        \endgl
\xe

Let's try to passivize: `The axe was used to cut the net (by the fisherman)'. The instrument will become the subject (we'll get back to that in argument structure). What will be the ordering of \gsc{Appl} and \gsc{Pass}?

\bigskip
\pex[exno=10]
    \a[label=b] \begingl
        \gla Nkhw\^{a}ngwa i-na-dúl-\textbf{ír}-\uline{idw}-á úk\=onde (ndí ms\=odzi)//
        \glb 9.axe 9\gsc{S}-\gsc{PAST}-cut-\gsc{APPL}-\gsc{PASS-FV} 14.net by 1.fisherman//
        \glft `The axe was used to cut the net (by the fisherman).'//
        \endgl
\xe

Schematic, ignoring the actual arguments:\\
\Tree
[.PassP
    [.Pass ]
    [.ApplP
        [.Appl ]
        \qroof{\emph{cut}}.vP
    ]
]    


Chi-Mwi:ni in \cite{baker85} from \cite{kisseberthabasheikh77}:
\pex[exno=56]   
    \a \begingl
        \gla Nu:ru Ø-chi-tes-ete chibu:ku//
        \glb Nuru \gsc{S-O}-bring-\gsc{asp} book//
        \glft `Nuru brought the book.'//
        \endgl
    \a \begingl
        \gla Nu:ru ø-m-tet-\textbf{el}-ele mwa:limu chibu:ku//
        \glb Nuru \gsc{S-O}-bring-\gsc{appl-asp} teacher book//
        \glft `Nuru brought the book to the teacher.'//
        \endgl
    \a \begingl
        \gla Mwa:limu ø-tet-\textbf{el}-el-\uline{a} chibu:ku na Nu:ru//
        \glb teacher \gsc{S}-bring-\gsc{appl-asp-pass} book by Nuru//
        \glft `The teacher was brought the book by Nuru.'//
        \endgl
\xe

Kinyarwanda in \cite{baker85} from \cite{kimenyi80}:
\pex[exno=57]
    \a \begingl
        \gla Umugabo a-ra-andik-a ibaruwa \textbf{n'i}-ikaramu//
        \glb man \gsc{S-pres}-write-\gsc{asp} letter with-pen//
        \glft `The man wrote [sic] the letter with the pen.'//
        \endgl
    \a \begingl
        \gla Umugabo a-ra-andik-\textbf{iish}-a ibaruwa ikaramu//
        \glb man \gsc{S-pres}-write-\gsc{instr-asp} letter pen//
        \glft `The man wrote the letter with the pen.'//
        \endgl
    \a \begingl
        \gla Ikaramu i-ra-andik-\textbf{iish}-\uline{w}-a ibaruwa n'umugabo//
        \glb pen \gsc{S-pres}-write-\gsc{instr-pass-asp} letter by-man//
        \glft `The pen was written-with [sic] the letter by the man.'//
        \endgl
    \a \begingl
        \gla Ibaruwa i-ra-andik-\textbf{iish}-\uline{w}-a ikaramu n'umugabo//
        \glb letter \gsc{S-pres}-write-\gsc{instr-pass-asp} pen by-man//
        \glft `The letter was written with the pen by the man.'//
        \endgl
\xe

        \subsubsection{Outside of Bantu}
Yupik \citep[43]{mithun99}:
\pex
    \a \begingl
        \gla ayag-ciq-\textbf{yugnarqe}-\uline{ni}-llru-u-q//
        \glb go-\gsc{FUT}-probably-claim-\gsc{PAST-INDIC.INTR}-3\gsc{SG}//
        \glft `He said he would probably go.'//
    \endgl
    \a \begingl
        \gla ayag-ciq-\uline{ni}-llru-\textbf{yugnarqe}-u-q//
        \glb go-\gsc{FUT}-claim-\gsc{PAST}-probably-\gsc{INDIC.INTR}-3\gsc{SG}//
        \glft `He probably said that he would go.'//
    \endgl
\xe

% \ex \begingl
%     \gla ayag-yug-umi-ite-qapiar-tu-a//
%     \glb go-want-be.in.state-not-really-\gsc{INDIC.INTR}-1\gsc{SG}//
%     \glft `I really don't want to go.'//
%     \endgl
% \xe
% (Why do we gloss these sometimes as words and sometimes as affixes? Because.)


Oji-Cree \citep{slavin05uot} in \cite{rice11}:
\pex[exno=11]
    \a \begingl
        \gla ishkwaa-niipaa-sookihpawn//
        \glb finish-at.night-be.snowing//
        \glft \longline{} // %`It stopped snowing at night.' (does not snow at night anymore)//
    \endgl
    \a \begingl
        \gla nipaa-ishkwaa-sookihpwan//
        \glb at.night-finish-be.snowing//
        \glft \longline{} // % `It stopped snowing at night.' (was snowing the whole day)//
    \endgl
\xe
\pex[exno=11']
    \a \begingl
        \gla kiimooci-kishahtapi-wiihsini//
        \glb secretly-fast-eat//
        \glft \longline{} // % `He secretly eats fast.' (nobody knows that he eats fast)//
    \endgl
    \a \begingl
        \gla kishahtapi-kiimooci-wiihsini//
        \glb fast-secretly-eat//
        \glft \longline{} // % `He eats secretly (nobody knows that he eats) and he does it fast.'//
    \endgl
\xe


Pulaar \citep{paster05}, with \gsc{com}prehensive \emph{id} `all' and \gsc{sep} \emph{it} which denotes the reverse of the action (so `open' + \gsc{sep} = `close').
\pex
    \a \begingl
        \gla mi udd-id-it-ii baafe ɗe fof//
        \glb 1\gsc{SG} close-\gsc{COM-SEP-past} door Det all//
        \glft `I opened [sic] all the doors (in sequence).'//
        \endgl
    \a \begingl
        \gla mi udd-it-id-ii baafe ɗe fof//
        \glb 1\gsc{SG} close-\gsc{SEP-COM-past} door Det all//
        \glft `I opened all the doors (at once).'//
        \endgl
\xe

Here, \gsc{rep}epetitive means `again'. Assume that one of the following means `make someone learn' and one means `teach' - which is which?
\pex
    \a \begingl
        \gla o jaŋŋg-in-it-ii kam//
        \glb 3\gsc{SG} learn-\gsc{CAUS-REP-past} 1\gsc{SG}//
        \glft \longline{} // % `He taught me again.' (taught me before)//
        \endgl
    \a \begingl
        \gla o jaŋŋg-it-in-ii kam//
        \glb 3\gsc{SG} learn-\gsc{REP-CAUS-past} 1\gsc{SG}//
        \glft \longline{} // % `‘He made me learn again.'//
        \endgl
\xe

    \subsection*{Interim summary}
\begin{itemize} \tightlist
    \item Rigid ordering for some (inflectional?) categories, e.g.~auxiliaries.
    \item Variable ordering depending on scope for some (inflectional?) categories.
    \item Always respect semantic and morphological scope.
\end{itemize}

Key references:
\cite{baker85} gets all the glory for coining the Mirror Principle but \cite{muysken81,muysken88} was there first. \cite{rice11} is an excellent overview of different factors in affix ordering, based in part on \cite{rice00}.


    \subsection{Longer chains}
Turkish \citep[368]{inkelasorgun98} 
\ex \begingl
	\glpreamble \emph{çekoslovakyalilaştiramayacaklarimizdanmiydiniz}//
	\gla çekoslovakya li laş tir ama yacak lar imiz dan mi ydi niz //
	\glb Czechoslovakia from become \gsc{CAUSE} unable Fut \gsc{PL} \gsc{1PL} \gsc{ABL} \gsc{INTERR} Past \gsc{2PL}//
	\glft `were you one of those whom we are not going to be able to turn into Czechoslovakians?'//
	\endgl
\xe

Japanese:
\ex \begingl
	\gla taro-ga kodomo-o \textbf{sodat-e-sase-rare-ta}//
	\glb Taro-\gsc{NOM} child-\gsc{ACC} rise-\gsc{CAUS}-\gsc{CAUS}-\gsc{PASS}-\gsc{PAST}//
	\glft `Taro was made to raise the child.'//
	\endgl
\xe

What would a structure for this look like?

\ex
% \Tree
% [.TP
% 	[.\emph{taro-ga} ]
% 	[.
% 		[.vP
% 			[.vP
% 				[.Ø ]
% 				[.
% 					[.vP
% 						[.\sout{\emph{taro-ga}} ]
% 						[.
% 							[.VP
% 								[.\emph{kodomo-o} ]		
% 								[.V \emph{sodat} ]
% 							]

\newpage
\bibliographystyle{linquiry2}
\bibliography{lingxbib}

\end{document}
