\documentclass[a4paper,12pt]{article}
%\usepackage[utf8]{inputenc}
\usepackage {fontspec} %,xunicode} %,xltxtra}
%	\setmainfont{Times New Roman}
\setmainfont[
Ligatures={TeX,Common},
PunctuationSpace=0,
Numbers={Proportional},
BoldFont = LibertinusSerif-Semibold.otf,
ItalicFont = LibertinusSerif-Italic.otf,
BoldItalicFont = LibertinusSerif-SemiboldItalic.otf,
BoldSlantedFont = LibertinusSerif-Semibold.otf,
SlantedFont    = LibertinusSerif-Regular.otf,
SlantedFeatures = {FakeSlant=0.25},
BoldSlantedFeatures = {FakeSlant=0.25},
SmallCapsFeatures = {FakeSlant=0},
]{LibertinusSerif-Regular.otf}

\usepackage[margin={1in}]{geometry}

\usepackage[dvipsnames]{xcolor}
\usepackage[linkcolor=purple,citecolor=ForestGreen,colorlinks=true,urlcolor=purple,pagebackref=false]{hyperref}

\usepackage{natbib}
\bibpunct[:]{(}{)}{;}{a}{}{,}
\setlength{\bibsep}{0pt plus 0.3ex}
\usepackage{graphicx}
\usepackage{multicol}
\usepackage{amsmath}
\usepackage{expex} %Linguistics examples
\lingset{aboveexskip=0.2ex,belowexskip=0.8ex,aboveglftskip=-0.2ex,interpartskip=0.2ex,labelwidth=!6pt,belowpreambleskip=0.1ex} %, *=*?}

\usepackage{mdwlist}  %
\usepackage[normalem]{ulem}
\usepackage{array} % >{\...} in tabulars
\setlength\parindent{0pt}

\usepackage{lastpage}
\usepackage{fancyhdr}
\addtolength{\headheight}{2.5pt}
\pagestyle{fancy} %or fancyplain
\lhead{Morphology 2024-25 (UoE)}
\rhead{Jigsaw group B}
\fancyfoot[C]{\thepage~of \pageref{LastPage}}
\fancypagestyle{plain}{%
\fancyhf{} % clear all header and footer fields
\fancyfoot[C]{Page \thepage~of \pageref{LastPage}} % except the center
\renewcommand{\headrulewidth}{0pt}
\renewcommand{\footrulewidth}{0pt}}

\definecolor{edir}{RGB}{193,0,67}
\definecolor{edib}{RGB}{0,50,95}
\newcommand\ruler{{\centering \color{edir} \rule[-0.5em]{0.6\columnwidth}{0.4pt}\par}}

\renewcommand\root[1]{$\sqrt{\text{#1}}$}

\usepackage{pifont}
\newcommand{\cmark}{\ding{51}}% 52
\newcommand{\xmark}{\ding{55}}%
\newcommand{\hand}{\ding{43}}

\newcommand\gsc[1]{\textsc{\lowercase{#1}}} %for glossing in small caps - comment out to return to caps

\makeatletter %Allow superscript ^ and subscript _
\catcode`_=\active%
\gdef_#1{\ensuremath{{}\sb{#1}}}%
\catcode`^=\active%
\gdef^#1{\ensuremath{{}\sp{#1}}}%
\makeatother

\title{Jigsaw group B}
\date{Morphology 2024-25, allomorphy}
\author{}

\begin{document}

In this activity we want to understand what morpheme is \textbf{triggering} (or \textbf{conditioning}) the allomorphy of what other morpheme. The same abstract relationship holds in all cases on this sheet. For each example, we want to look at the alternations and figure out which morpheme is influencing which one; and also, on what information the alternation depends. % syntactic, inward-looking

\bigskip
What does the form of the definite article suffix in Bulgarian depend on? What would the \textbf{tree structure} (or bracket structure) for this interaction look like? Which node is looking at which? \textbf{What kind} of information is it looking for?


\begin{multicols}{2}
\pex
    \a gaz\\
    gas.\gsc{MASC}\\
    `gas (state of matter)'
     \a gaz-ta\\
        gas.\gsc{FEM}-\gsc{DEF}\\
        `the gas (fuel)'
\xe
\pex
    \a gaz\\
    gas.\gsc{FEM}\\
    `gas (fuel)'
   \a gaz-a\\
    gas.\gsc{MASC}-\gsc{DEF}\\
    `the gas (state of matter)' 
\xe
\end{multicols}
\pex
    \a kolyano\\
    knee\\
    `knee'
    \a kolene\\
    knee.\gsc{PL}\\
    `knees'
    \a kolene-te\\
    knee.\gsc{PL}-\gsc{DEF}\\
    `the knees'
\xe

\bigskip

Here is a similar pattern for the form of Latin tense:

\pex \label{ex:perf-agr-past}
\a \begingl
\gla am-\=a-\underline{ba}-m//
\glb \root{am}-\gsc{TH}-\underline{Past}-\gsc{1SG}//
\glft `I loved'//
\endgl

\a \begingl
\gla am-\=a-\fbox{ve}-\underline{ra}-m//
\glb \root{am}-\gsc{TH}-\fbox{Perf}-\underline{Past}-\gsc{1SG}//
\glft `I had loved'//
\endgl
\xe


\pex \label{ex:perf-agr-fut}
\a \begingl
\gla am-\=a-\underline{b}-\=o//
\glb \root{am}-\gsc{TH}-\underline{Fut}-\gsc{1SG}//
\glft `I will love'//
\endgl

\a \begingl
\gla am-\=a-\fbox{ve}-\underline{r}-\=o//
\glb \root{am}-\gsc{TH}-\fbox{Perf}-\underline{Fut}-\gsc{1SG}//
\glft `I will have loved'//
\endgl
\xe

\bigskip

And we could even think of English irregular plurals this way (focus on the form of the plural morpheme, not the root):
\pex
	\a sheep, deer
	\a oxen
	\a \dots{}
	\a dogs, cats
\xe

\vfill

If you finish with time to spare: what other interactions would you expect to exist?

\end{document}