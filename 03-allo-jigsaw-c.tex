\documentclass[a4paper,12pt]{article}
%\usepackage[utf8]{inputenc}
\usepackage {fontspec} %,xunicode} %,xltxtra}
%	\setmainfont{Times New Roman}
\setmainfont[
Ligatures={TeX,Common},
	PunctuationSpace=0,
	Numbers={Proportional},
	BoldFont = LibertinusSerif-Semibold.otf,
	ItalicFont = LibertinusSerif-Italic.otf,
	BoldItalicFont = LibertinusSerif-SemiboldItalic.otf,
	BoldSlantedFont = LibertinusSerif-Semibold.otf,
	SlantedFont    = LibertinusSerif-Regular.otf,
	SlantedFeatures = {FakeSlant=0.25},
	BoldSlantedFeatures = {FakeSlant=0.25},
	SmallCapsFeatures = {FakeSlant=0},
]{LibertinusSerif-Regular.otf}

\usepackage[margin={1in}]{geometry}

\usepackage[dvipsnames]{xcolor}
\usepackage[linkcolor=purple,citecolor=ForestGreen,colorlinks=true,urlcolor=purple,pagebackref=false]{hyperref}

\usepackage{natbib}
	\bibpunct[:]{(}{)}{;}{a}{}{,}
	\setlength{\bibsep}{0pt plus 0.3ex}
\usepackage{graphicx}
\usepackage{multicol}
\usepackage{amsmath}
\usepackage{expex} %Linguistics examples
\lingset{aboveexskip=0.2ex,belowexskip=0.8ex,aboveglftskip=-0.2ex,interpartskip=0.2ex,labelwidth=!6pt,belowpreambleskip=0.1ex} %, *=*?}

\usepackage{mdwlist}  %
\usepackage[normalem]{ulem}
\usepackage{array} % >{\...} in tabulars
\setlength\parindent{0pt}

\usepackage{lastpage}
\usepackage{fancyhdr}
	\addtolength{\headheight}{2.5pt}
	\pagestyle{fancy} %or fancyplain
	\lhead{Morphology 2024-25 (UoE)}
	\rhead{Jigsaw group C}
	\fancyfoot[C]{\thepage~of \pageref{LastPage}}
	\fancypagestyle{plain}{%
		\fancyhf{} % clear all header and footer fields
		\fancyfoot[C]{Page \thepage~of \pageref{LastPage}} % except the center
		\renewcommand{\headrulewidth}{0pt}
		\renewcommand{\footrulewidth}{0pt}}

\definecolor{edir}{RGB}{193,0,67}
\definecolor{edib}{RGB}{0,50,95}
\newcommand\ruler{{\centering \color{edir} \rule[-0.5em]{0.6\columnwidth}{0.4pt}\par}}

\renewcommand\root[1]{$\sqrt{\text{#1}}$}

\usepackage{pifont}
	\newcommand{\cmark}{\ding{51}}% 52
	\newcommand{\xmark}{\ding{55}}%
	\newcommand{\hand}{\ding{43}}

\newcommand\gsc[1]{\textsc{\lowercase{#1}}} %for glossing in small caps - comment out to return to caps

\makeatletter %Allow superscript ^ and subscript _
\catcode`_=\active%
\gdef_#1{\ensuremath{{}\sb{#1}}}%
\catcode`^=\active%
\gdef^#1{\ensuremath{{}\sp{#1}}}%
\makeatother

\title{Jigsaw group C}
\date{Morphology 2024-25, allomorphy}
\author{}

\begin{document}

In this activity we want to understand what morpheme is \textbf{triggering} (or \textbf{conditioning}) the allomorphy of what other morpheme. The same abstract relationship holds in all cases on this sheet. For each example, we want to look at the alternations and figure out which morpheme is influencing which one; and also, on what information the alternation depends. % phonological, inward-looking

\bigskip
The same kind of interaction is happening in~(\nextx) and also in~(\anextx). What would the \textbf{tree structure} (or bracket structure) for this interaction look like? Which node is looking at which? \textbf{What kind} of information is it looking for?


\pex \label{ex:past-en}
	\a \emph{ungrade}\textbf{[əd]} 
	\a \emph{jam}\textbf{[d] }
	\a  \emph{jump}\textbf{[t] }
\xe

\pex
\a   \emph{\textbf{a} dog}		 
\a  \emph{\textbf{an} apple} 
\xe


\bigskip

What would the \textbf{tree structure} for this interaction look like? Which node is looking at which?
\vspace{3cm}



Similar phenomenon in Moroccan Arabic (and other dialects):
\ex
\begin{tabular}{l>{\em}ll>{\em}ll}
	a.& xtʕa-h & `his error'  & ktab-u & `his book'\\
	b.& ʃafu-h & `they saw him' & ʃaf-u & `he saw him'\\
	c.& mʕa-h & `with him' & menn-u & `from him'\\
\end{tabular}
\xe


\bigskip
And in Tahitian:
\ex
\begin{tabular}{l>{\em}ll>{\em}ll}
	a.& 'amu & `eat' & \textbf{fa'a}-'amu & `make eat'\\
	b.& rave & `do, make' & \textbf{fa'a}-rave & `make make'\\
	c.& tai'o & `read' & \textbf{fa'a}-tai'o & `make read'\\\hline
	d.& mana'o & `think' & \textbf{ha'a}-mana'o & `remember'\\
	e.& fiu & `grow tired' & \textbf{ha'a}-fiu & `be bored'\\
	f.& veve & `be poor' & \textbf{ha'a}-veve & `impoverish'\\
\end{tabular}
\xe

\bigskip

And also in Chaha:
\ex Imperative\\
\begin{tabular}{lll}
	2\gsc{SG.M} & 2\gsc{SG.F} & gloss \\\hline
	nomæd & nomædʲ & `love' \\
	noqo\d{t} & noqo\d{t}ʲ & `kick'\\
	goræz & goræzʲ & `be old'\\
\end{tabular}
\xe
% \ex Perfective\\
% \begin{tabular}{lll}
% 	without object & with object & gloss \\\hline
% 	qænæf & qænæfʷ & `knock down'\\
% 	nækæb & nækæbʷ & `find'\\\hline
% 	nækæs & nækʷæs & `bite' \\
% 	kæfæt & kæfʷæt & `open' \\\hline
% 	qæ\d{t}ær & qʷæ\d{t}ær & `kill'\\
% 	mæsær & mʷæsær & `seem'\\
% \end{tabular}
% \xe

\vfill

If you finish with time to spare: what other interactions would you expect to exist?

\end{document}