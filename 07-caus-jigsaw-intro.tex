\documentclass[a4paper,12pt]{article}
%\usepackage[utf8]{inputenc}
\usepackage {fontspec} %,xunicode} %,xltxtra}
%	\setmainfont{Times New Roman}
\setmainfont[
Ligatures={TeX,Common},
PunctuationSpace=0,
Numbers={Proportional},
BoldFont = LibertinusSerif-Semibold.otf,
ItalicFont = LibertinusSerif-Italic.otf,
BoldItalicFont = LibertinusSerif-SemiboldItalic.otf,
BoldSlantedFont = LibertinusSerif-Semibold.otf,
SlantedFont    = LibertinusSerif-Regular.otf,
SlantedFeatures = {FakeSlant=0.25},
BoldSlantedFeatures = {FakeSlant=0.25},
SmallCapsFeatures = {FakeSlant=0},
]{LibertinusSerif-Regular.otf}

\usepackage[margin={1in}]{geometry}

\usepackage[dvipsnames]{xcolor}
\usepackage[linkcolor=purple,citecolor=ForestGreen,colorlinks=true,urlcolor=purple,pagebackref=false]{hyperref}
\providecommand{\tightlist}{%
	\setlength{\itemsep}{0pt}\setlength{\parskip}{0pt}}

\usepackage{natbib}
\bibpunct[:]{(}{)}{;}{a}{}{,}
\setlength{\bibsep}{0pt plus 0.3ex}
\usepackage{graphicx}
\usepackage{multicol}
\usepackage{amsmath}
\usepackage{expex} %Linguistics examples
\lingset{aboveexskip=0.2ex,belowexskip=0.8ex,aboveglftskip=-0.2ex,interpartskip=0.2ex,labelwidth=!6pt,belowpreambleskip=0.1ex} %, *=*?}

\usepackage[nocenter]{qtree} % trees

\usepackage{mdwlist}  %
\usepackage[normalem]{ulem}
\usepackage{array} % >{\...} in tabulars
\setlength\parindent{0pt}
\usepackage{dashrule} %\hdashrule

\usepackage{lastpage}
\usepackage{fancyhdr}
\addtolength{\headheight}{2.5pt}
\pagestyle{fancy} %or fancyplain
\lhead{Morphology 2024-25 (UoE)}
\rhead{Reverse jigsaw}
\fancyfoot[C]{\thepage~of \pageref{LastPage}}
\fancypagestyle{plain}{%
\fancyhf{} % clear all header and footer fields
\fancyfoot[C]{Page \thepage~of \pageref{LastPage}} % except the center
\renewcommand{\headrulewidth}{0pt}
\renewcommand{\footrulewidth}{0pt}}

\definecolor{edir}{RGB}{193,0,67}
\definecolor{edib}{RGB}{0,50,95}
\newcommand\ruler{{\centering \color{edir} \rule[-0.5em]{0.6\columnwidth}{0.4pt}\par}}

\renewcommand\root[1]{$\sqrt{\text{#1}}$}

\usepackage{pifont}
\newcommand{\cmark}{\ding{51}}% 52
\newcommand{\xmark}{\ding{55}}%
\newcommand{\hand}{\ding{43}}

\newcommand\gsc[1]{\textsc{\lowercase{#1}}} %for glossing in small caps - comment out to return to caps

\makeatletter %Allow superscript ^ and subscript _
\catcode`_=\active%
\gdef_#1{\ensuremath{{}\sb{#1}}}%
\catcode`^=\active%
\gdef^#1{\ensuremath{{}\sp{#1}}}%
\makeatother

\title{Reverse jigsaw} % transitive, verb-derived, cause O to be V-ed, relating causing and caused events
\date{Morphology 2024-25, causatives}
\author{}

\begin{document}

This jigsaw puzzle examines different kinds of \emph{causatives}. For simplicity, assume that we have one morpheme \gsc{CAUS} in various languages -- the exact syntactic label is not important right now. You can think of it as the verbalizer v, or as its own morpheme \gsc{CAUS}.

\bigskip
We're practising a new skill this time: taking our theory and using it to make predictions. Assuming a simple transitive structure like~(\nextx), we'll ask ourselves:
\begin{itemize} \tightlist
		\item[Q1] Where could \gsc{CAUS} be merged (where could we insert it)?
		\item[Q2] What would the meaning be?
\end{itemize}

\ex Theo feared the cat.\\
	\Tree
	[.VoiceP
		[.DP\\\emph{Theo} ]
		[.
			[.Voice ] 
			[.vP
				[.v
					[.\root{\gsc{FEAR}} ]
					[.v ]
				]
				[.DP\\\emph{the cat} ]
			]
		]
	]	
\xe

We'll look at three possible positions, which we'll call ``middle'', ``high'', and ``low''. In the next phase of this \emph{reverse jigsaw}, each group will receive a sheet with some data and try to match it up with one of these structures. How close will you be with your predictions?

\subsection*{``Middle'' causative}

One option is that \gsc{CAUS} attaches above the vP, in a clause that's otherwise similar to~(\nextx), but which might need a different verb and arguments. What would be a possible example? What would the meaning be?

\ex
        \Tree
		[.
			[.{DP1} ]
			[.
				[.Voice ]
				[.
					[.\gsc{CAUS} ]
					[.vP
						[.v
							[.\root{\gsc{root}} ]
							[.v ]
						]
						[.{DP2} ]
					]
				]
			]
		]
\xe

\newpage
\subsection*{``High'' causative}

We could also embed the entire VoiceP under \gsc{CAUS}. Again, what would the meaning be?

\ex
        \Tree
			[.VoiceP
				[.{DP1} ]
				[.
					[.Voice ]
					[.
						[.\gsc{CAUS} ]
						[.VoiceP
						[.{DP2} ]
						[.
						[.Voice ]
						[.vP 
						[.v
						[.\root{\gsc{ROOT}} ]
						[.v ]
						]
						[.{(DP3)} ]
						]
						]
						]
					]
				]
			]
\xe

\subsection*{``Low'' causative}

Lastly, \gsc{CAUS} could merge very low, either before v or instead of it. What could the effect here be like?
\ex
    \Tree
	[.VoiceP
		[.DP ]
		[.
			[.Voice ]
			[.vP
				[.v
					[.
						[.\root{\gsc{ROOT}} ]
						[.{\gsc{CAUS}} ]					
					]
					[.v ]
				]
				[.DP ]
			]
		]
]
\xe

\end{document}