\documentclass[a4paper,12pt]{article}

%\usepackage{times}
%\usepackage[T3,T1]{fontenc}
%\usepackage[utf8x]{inputenc}
% \usepackage[T1]{fontenc}
% \usepackage{mathptmx}  

%\usepackage{fontspec} %,xunicode} % xltxtra
 %   \defaultfontfeatures{Ligatures=TeX}
 %   \setmainfont{Brill}[
 %   Extension=.ttf,
  %  UprightFont=*-Roman,
  %  BoldFont=*-Bold,
  %  ItalicFont=*-Italic,
 %   BoldItalicFont=*-Bold-Italic,
  %  Renderer=ICU]
% 	\usefonttheme{serif}
%	\setmainfont[Renderer=ICU]{Charis SIL}
% 	\setmainfont[Renderer=ICU]{Brill}

\usepackage{amsmath,amssymb} % for $\text{}$
\usepackage{url,natbib} 
\usepackage[dvipsnames]{xcolor} %,bm}
\usepackage{geometry,vmargin,setspace}
\usepackage{multirow}
\setmarginsrb{1in}{1in}{1in}{1in}{13.6pt}{0.1in}{0.1in}{0.2in}
% pdflscape} %rotating
    \setlength{\parindent}{0pt}

\usepackage{stmaryrd}
\usepackage{wasysym} %checkbox
\usepackage[normalem]{ulem} % for \sout{}
% \usepackage{mdwlist} % for \begin{itemize*} - but prefer \tightlist
\providecommand{\tightlist}{%
	\setlength{\itemsep}{0pt}\setlength{\parskip}{0pt}}

\usepackage{natbib}
	\bibpunct[:]{(}{)}{;}{a}{}{,}
	\setlength{\bibsep}{0pt plus 0.3ex}

\usepackage{longtable}
\usepackage[linkcolor=purple,citecolor=ForestGreen,colorlinks=true,urlcolor=gray,pagebackref=true]{hyperref}
\usepackage{multicol}
\usepackage{dashrule} %\hdashrule
\usepackage{array} % >{\...} in tabulars
\usepackage{arydshln}

\usepackage[framemethod=tikz,footnoteinside=false]{mdframed} 

\usepackage{expex} %Linguistics examples
\lingset{aboveexskip=0ex,belowexskip=0.5ex,aboveglftskip=-0.3em,interpartskip=0ex,labelwidth=!6pt,belowpreambleskip=0.1ex} %, *=*?}
%	\lingset{aboveexskip=0.5ex,belowexskip=0.5ex,*=??}

\usepackage[nocenter]{qtree} % trees
\usepackage{tikz}
    \tikzstyle{every picture}+=[remember picture]

\usepackage{tipa} % convenient for \textsubarch etc.

\newcommand\trace{\rule[-0.5ex]{0.5cm}{.4pt}}
\newcommand\midline{\rule[-0.5ex]{4cm}{.4pt}}
\newcommand\longline{\rule[-0.5ex]{7cm}{.4pt}}

\bibpunct[:]{(}{)}{;}{a}{}{,}
	\setlength{\bibsep}{0pt plus 0.3ex}

\usepackage{pifont}
\newcommand{\cmark}{\ding{51}}% 52
\newcommand{\xmark}{\ding{55}}%
 \newcommand{\hand}{\ding{43}}

\newcommand\zero{\O{}}
\newcommand\itp[1]{\textit{\textipa{#1}}}
\newcommand\gsc[1]{\textsc{\lowercase{#1}}} %for glossing in small caps - comment out to return to caps
\renewcommand\root[1]{$\sqrt{\text{#1}}$}

\newcommand\blue[1]{\textcolor{blue}{#1}}
\newcommand\red[1]{\textcolor{red}{#1}}
\newcommand\green[1]{\textcolor{ForestGreen}{#1}}
\newcommand\gray[1]{\textcolor{gray}{#1}}
\newcommand\denote[1]{$\llbracket$#1$\rrbracket$}
\newcommand\lra{$\leftrightarrow$}

\newcommand\vz{\text{Voice$_{\text{\{--D\}}}$}}
\newcommand\vd{\text{Voice$_{\text{\{+D\}}}$}}
\newcommand\pz{\text{$p_{\text{\zero}}$}}
\newcommand\va{\root{\gsc{ACTION}}}
\newcommand{\tkal}{\emph{XaYaZ}}
\newcommand{\tpie}{\emph{XiY̯eZ}}
\newcommand{\tpua}{\emph{XuY̯aZ}}
\newcommand{\thif}{\emph{heXYiZ}}
\newcommand{\thuf}{\emph{huXYaZ}}
\newcommand{\thit}{\emph{hitXaY̯eZ}}
\newcommand{\tnif}{\emph{niXYaZ}}
\newcommand\dgs[1]{\textsubarch{#1}}
\newcommand\del[1]{\sout{#1}}

\makeatletter %Allow superscript ^ and subscript _
\catcode`_=\active%
\gdef_#1{\ensuremath{{}\sb{#1}}}%
\catcode`^=\active%
\gdef^#1{\ensuremath{{}\sp{#1}}}%
\makeatother

\begin{document}
% \pagenumbering{gobble}
% \singlespacing


\hfill \emph{Théories linguistiques 2024, Utrecht}


\section{Ordre des affixes}

    \subsection{Auxiliaires et échauffement}
Dans quel ordre apparaissent les auxiliaires en français?

\pex
    \a Elle va avoir eu gagn-é la course.
    \a \ljudge{*} Elle a eu va gagn-er la course.
\xe
\pex
    \a Le gâteau va avoir été mangé.
    \a \ljudge{*} Le gâteau a eté avoir aller être mangé.
\xe

\midline{}

\bigskip
Comment s'imbriquent les éléments entre eux:
% \begin{itemize} \tightlist
%     \item Takes semantic \emph{scope} over the next: the future of the perfect of the passive of an eating event.
%     \item Takes morpho(-phono)logical \emph{scope} over the next: \emph{have}_{\gsc{perf}} triggers participial morphology on \emph{eat-en}, \emph{is}_{\gsc{prog}} triggers progressive morphology on \emph{winn-ing}, etc.
% \end{itemize}

\vspace{3cm}

\bigskip
Représentons cela formellement:

\newpage
\ex
% \Tree
% [.TP
%     [.\emph{The cake} ]
%     [.
%         [.T\\\emph{will} ]
%         [.PerfP/AspP
%             [.Perf\\\emph{have} ]
%             [.ProgP
%                 [.Prog\\\emph{been} ]
%                 [.PassP
%                     [.Pass\\\emph{being} ]
%                     [.VP\\\emph{eaten} ]
%                 ]
%             ]
%         ]
%     ]
% ]
\xe
\vspace{3cm}

Quel est l'ordre observé en latin \citep{embick10,kastnerzu17,kastner18nllt}?
\pex
    \a \begingl
        \gla am-\=a-ve-ra-m//
        \glb \root{AIM}-\gsc{THÈME}-Perf-Passé-1\gsc{SG}//
        \glft `J'ai eu aimé'//
        \endgl
    \a \begingl
        \gla am-\=a-ve-r-\=o//
        \glb \root{AIM}-\gsc{THÈME}-Perf-Fut-1SG//
        \glft `Je vais avoir aimé'//
        \endgl
\xe

\vspace{3cm}

% V < Th < Perf < T.
% \begin{itemize} \tightlist
%     \item $\approx$ T > Perf > V.
%     \item Like in English, except in morphology rather than syntax.
%     \item Still get morphological conditioning, but we'll return to that later (allomorphy).
% \end{itemize}

En quoi cela est-il différent du français?

\longline{}

    \subsection{Le Principe du Miroir}
        \subsubsection{Introduction}
Dans quel ordre apparaissent les suffixes réciproque et causatif en Chiche\^{w}a \citep{alsina99}?
\ex[exno=3] \begingl
    \gla Al\=enje a-na-mény-\textbf{án}-\uline{its}-á mb\^{u}zi//
    \glb 2.chasseurs 2\gsc{s}-\gsc{PASSÉ}-frapper-\gsc{RECIP}-\gsc{CAUS}-\gsc{V-fin} 10.boucs//
    \glft `Les boucs, les chasseurs les ont fait se frapper l'un l'autre.'//
    \endgl
\xe
\ex[exno=4] \begingl
    \gla Al\=enje a-na-mény-\uline{éts}-\textbf{an}-a mb\^{u}zi//
    \glb 2.chasseurs \gsc{2S}-\gsc{PASSÉ}-frapper-\gsc{CAUS}-\gsc{RECIP}-\gsc{FV} 10.boucs//
    \glft `Les chasseurs se sont fait l'un l'autre frapper les boucs.'//
    \endgl
\xe

\longline{}
% The order is whatever it needs to be to get scope right; You go from the verb ``outwards''.

\bigskip
Voici une autre paire \citep{hymanmchombo92}:
\pex
    \a mang-\textbf{an}-\uline{its}\\
    attacher-\gsc{RECIP}-\gsc{CAUS}\\
    `faire s'attacher l'un l'autre'
    \a mang-\uline{its}-\textbf{an}\\
    attacher-\gsc{CAUS}-\gsc{RECIP}\\
    `se faire l'un l'autre attacher'
\xe

À quoi ressemblerait cette langue  (\emph{chichewa'}) si elle avait des préfixes plutôt que des suffixes? 
\pex
    \a \midline{} % its-an-mang
    \a \midline{} % an-its-mang
\xe

% Because hierarchy is what matters (relative ordering), rather than linear order.

\bigskip

Un autre example: le bemba tel que décrit par \cite{baker85}, qui cite \cite{givon76}.
\pex[exno=49]
    \a \begingl
        \gla Naa-mon-an-ya Mwape na Mutumba//
        \glb 1s.\gsc{S}-\gsc{passé}-voir-\gsc{recip-caus} Mwape et Mutumba//
        \glft `Mwape et Mutumba, je les ai fait se voir l'un l'autre.'//
        \endgl
    \a \begingl
        \gla Mwape na Chilufya baa-mon-eshy-ana Mutumba//
        \glb Mwape et Chilufya 3p.\gsc{S}-voir-\gsc{caus-recip} Mutumba//
        \glft `Mwape et Chilufya se sont fait l'un l'autre voir  Mutumba.'//
        \endgl
\xe

        \subsubsection{Passivation}
En chichewa, l'applicatif peut être utilisé pour les instruments \citep{alsina99}:
\pex[exno=10] 
    \a \begingl
        \gla Ms\=odzi a-na-dúl-\textbf{ír}-a \textbf{nkhw\^{a}ngwa} uk\=onde//
        \glb 1.fisherman \gsc{1S}-\gsc{PASSÉ}-couper-\gsc{APPL-VF} 9.hache 14.filet//
        \glft `Le pêcheur a coupé le filet avec une hache.'//
        \endgl
\xe

Essayons de passiver de cette manière: `La hache a été utilisée pour couper le filet (par le pêcheur). L'instrument va devenir le sujet. Quel va être l'ordre de  \gsc{Appl} et \gsc{Pass}?

\bigskip
\pex[exno=10]
    \a[label=b] \begingl
        \gla Nkhw\^{a}ngwa i-na-dúl-\textbf{ír}-\uline{idw}-á úk\=onde (ndí ms\=odzi)//
        \glb 9.hache 9\gsc{S}-\gsc{PASSÉ}-couper-\gsc{APPL}-\gsc{PASS-VF} 14.filet par 1.pêcheur//
        \glft `La hache a été utilisée pour couper le filet (par le pêcheur).'//
        \endgl
\xe

Schématiquement, en ignorant les arguments:\\
\Tree
[.PassP
    [.Pass ]
    [.ApplP
        [.Appl ]
        \qroof{\emph{couper}}.vP
    ]
]    


Le chi-Mwi:ni dans \cite{baker85}, données de \cite{kisseberthabasheikh77}:
\pex[exno=56]   
    \a \begingl
        \gla Nu:ru Ø-chi-tes-ete chibu:ku//
        \glb Nuru \gsc{S-O}-apporter-\gsc{asp} livre//
        \glft `Nuru a apporté le livre.'//
        \endgl
    \a \begingl
        \gla Nu:ru ø-m-tet-\textbf{el}-ele mwa:limu chibu:ku//
        \glb Nuru \gsc{S-O}-apporter-\gsc{appl-asp} professeur livre//
        \glft `Nuru a apporté le livre au professeur.'//
        \endgl
    \a \begingl
        \gla Mwa:limu ø-tet-\textbf{el}-el-\uline{a} chibu:ku na Nu:ru//
        \glb professeur \gsc{S}-apporter-\gsc{appl-asp-pass} livre par Nuru//
        \glft Littéralement `Le professeur a été apporté le livre par Nuru (en bon français: le livre a été apporté au professeur par Nuru).'//
        \endgl
\xe

Le kinyarwanda dans \cite{baker85}, données de  \cite{kimenyi80}:
\pex[exno=57]
    \a \begingl
        \gla Umugabo a-ra-andik-a ibaruwa \textbf{n'i}-ikaramu//
        \glb homme \gsc{S-prés}-écrire-\gsc{asp} lettre avec-stylo//
        \glft `L'homme a écrit  [sic] la lettre avec le stylo.'//
        \endgl
    \a \begingl
        \gla Umugabo a-ra-andik-\textbf{iish}-a ibaruwa ikaramu//
        \glb homme \gsc{S-prés}-écrire-\gsc{instr-asp} lettre stylo//
        \glft `L'homme a écrit  [sic] la lettre avec le stylo.'//
        \endgl
    \a \begingl
        \gla Ikaramu i-ra-andik-\textbf{iish}-\uline{w}-a ibaruwa n'umugabo//
        \glb stylo \gsc{S-prés}-écrire-\gsc{instr-pass-asp} lettre par-homme//
        \glft `Le stylo a été écrit-avec  [sic] la lettre par l'homme.'//
        \endgl
    \a \begingl
        \gla Ibaruwa i-ra-andik-\textbf{iish}-\uline{w}-a ikaramu n'umugabo//
        \glb lettre \gsc{S-prés}-écrire-\gsc{instr-pass-asp} stylo par-homme//
        \glft `La lette a été écrite avec le stylo par l'homme.'//
        \endgl
\xe

        \subsubsection{Ailleurs qu'en bantou}
Yupik \citep[43]{mithun99}:
\pex
    \a \begingl
        \gla ayag-ciq-\textbf{yugnarqe}-\uline{ni}-llru-u-q//
        \glb aller-\gsc{FUT}-probablement-dire-\gsc{PASSÉ-INDIC.INTR}-3\gsc{SG}//
        \glft `Il a dit qu'il irait probablement.'//
    \endgl
    \a \begingl
        \gla ayag-ciq-\uline{ni}-llru-\textbf{yugnarqe}-u-q//
        \glb aller-\gsc{FUT}-dire-\gsc{PAST}-probablement-\gsc{INDIC.INTR}-3\gsc{SG}//
        \glft `Il a probablement dit qu'il irait.'//
    \endgl
\xe

% \ex \begingl
%     \gla ayag-yug-umi-ite-qapiar-tu-a//
%     \glb go-want-be.in.state-not-really-\gsc{INDIC.INTR}-1\gsc{SG}//
%     \glft `I really don't want to go.'//
%     \endgl
% \xe
% (Why do we gloss these sometimes as words and sometimes as affixes? Because.)


Oji-Cree \citep{slavin05uot} dans \cite{rice11}:
\pex[exno=11]
    \a \begingl
        \gla ishkwaa-niipaa-sookihpawn//
        \glb finir-dans.la.nuit-être.neigeant//
        \glft \longline{} // %`It stopped snowing at night.' (does not snow at night anymore)//
    \endgl
    \a \begingl
        \gla nipaa-ishkwaa-sookihpwan//
        \glb dans.la.nuit-finir-être.neigeant//
        \glft \longline{} // % `It stopped snowing at night.' (was snowing the whole day)//
    \endgl
\xe
\pex[exno=11']
    \a \begingl
        \gla kiimooci-kishahtapi-wiihsini//
        \glb secrètement-vite-manger//
        \glft \longline{} // % `He secretly eats fast.' (nobody knows that he eats fast)//
    \endgl
    \a \begingl
        \gla kishahtapi-kiimooci-wiihsini//
        \glb vite-secrètement-manger//
        \glft \longline{} // % `He eats secretly (nobody knows that he eats) and he does it fast.'//
    \endgl
\xe


    \subsection*{Résumé}
\begin{itemize} \tightlist
    \item Certaines catégories suivent un ordre rigide, come par exemple les auxiliaires.
    \item D'autres catégories suivent un ordre variable qui dépend de la \textit{portée} sémantique.
    \item La portée morphologique reflète systèmatiquement la portée sémantique.
\end{itemize}

Références clé:
\cite{baker85} a trouvé la formulation du principle du miroir, mais  \cite{muysken81,muysken88} est le premier à avoir trouvé la généralisation. 

  

\newpage
\bibliographystyle{linquiry2}
\bibliography{lingxbib}

\end{document}
