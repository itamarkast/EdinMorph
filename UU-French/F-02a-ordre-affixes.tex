\documentclass[a4paper,12pt]{article}

%\usepackage{times}
%\usepackage[T3,T1]{fontenc}
%\usepackage[utf8x]{inputenc}
% \usepackage[T1]{fontenc}
% \usepackage{mathptmx}  

%\usepackage{fontspec} %,xunicode} % xltxtra
 %   \defaultfontfeatures{Ligatures=TeX}
 %   \setmainfont{Brill}[
 %   Extension=.ttf,
  %  UprightFont=*-Roman,
  %  BoldFont=*-Bold,
  %  ItalicFont=*-Italic,
 %   BoldItalicFont=*-Bold-Italic,
  %  Renderer=ICU]
% 	\usefonttheme{serif}
%	\setmainfont[Renderer=ICU]{Charis SIL}
% 	\setmainfont[Renderer=ICU]{Brill}

\usepackage[solution, spaces]{myProbSol} %   Optional arguments:
%                                               'solution': reveals solutions
%                                               'spaces': leaves whitespace corresponding to the size of the solutions (has no effect if you also pass 'solution')


\usepackage{amsmath,amssymb} % for $\text{}$
\usepackage{url,natbib} 
\usepackage[dvipsnames]{xcolor} %,bm}
\usepackage{geometry,vmargin,setspace}
\usepackage{multirow}
\setmarginsrb{1in}{1in}{1in}{1in}{13.6pt}{0.1in}{0.1in}{0.2in}
% pdflscape} %rotating
    \setlength{\parindent}{0pt}

\usepackage{stmaryrd}
\usepackage{wasysym} %checkbox
\usepackage[normalem]{ulem} % for \sout{}
% \usepackage{mdwlist} % for \begin{itemize*} - but prefer \tightlist
\providecommand{\tightlist}{%
	\setlength{\itemsep}{0pt}\setlength{\parskip}{0pt}}

\usepackage{natbib}
	\bibpunct[:]{(}{)}{;}{a}{}{,}
	\setlength{\bibsep}{0pt plus 0.3ex}

\usepackage{longtable}
\usepackage[linkcolor=purple,citecolor=ForestGreen,colorlinks=true,urlcolor=gray,pagebackref=true]{hyperref}
\usepackage{multicol}
\usepackage{dashrule} %\hdashrule
\usepackage{array} % >{\...} in tabulars
\usepackage{arydshln}

\usepackage[framemethod=tikz,footnoteinside=false]{mdframed} 

\usepackage{expex} %Linguistics examples
\lingset{aboveexskip=0ex,belowexskip=0.5ex,aboveglftskip=-0.3em,interpartskip=0ex,labelwidth=!6pt,belowpreambleskip=0.1ex} %, *=*?}
%	\lingset{aboveexskip=0.5ex,belowexskip=0.5ex,*=??}

\usepackage[nocenter]{qtree} % trees
\usepackage{tikz}
    \tikzstyle{every picture}+=[remember picture]

\usepackage{tipa} % convenient for \textsubarch etc.

\newcommand\trace{\rule[-0.5ex]{0.5cm}{.4pt}}
\newcommand\midline{\rule[-0.5ex]{4cm}{.4pt}}
\newcommand\longline{\rule[-0.5ex]{7cm}{.4pt}}

\bibpunct[:]{(}{)}{;}{a}{}{,}
	\setlength{\bibsep}{0pt plus 0.3ex}

\usepackage{pifont}
\newcommand{\cmark}{\ding{51}}% 52
\newcommand{\xmark}{\ding{55}}%
 \newcommand{\hand}{\ding{43}}

\newcommand\zero{\O{}}
\newcommand\itp[1]{\textit{\textipa{#1}}}
\newcommand\gsc[1]{\textsc{\lowercase{#1}}} %for glossing in small caps - comment out to return to caps
\renewcommand\root[1]{$\sqrt{\text{#1}}$}

\newcommand\blue[1]{\textcolor{blue}{#1}}
\newcommand\red[1]{\textcolor{red}{#1}}
\newcommand\green[1]{\textcolor{ForestGreen}{#1}}
\newcommand\gray[1]{\textcolor{gray}{#1}}
\newcommand\denote[1]{$\llbracket$#1$\rrbracket$}
\newcommand\lra{$\leftrightarrow$}

\newcommand\vz{\text{Voice$_{\text{\{--D\}}}$}}
\newcommand\vd{\text{Voice$_{\text{\{+D\}}}$}}
\newcommand\pz{\text{$p_{\text{\zero}}$}}
\newcommand\va{\root{\gsc{ACTION}}}
\newcommand{\tkal}{\emph{XaYaZ}}
\newcommand{\tpie}{\emph{XiY̯eZ}}
\newcommand{\tpua}{\emph{XuY̯aZ}}
\newcommand{\thif}{\emph{heXYiZ}}
\newcommand{\thuf}{\emph{huXYaZ}}
\newcommand{\thit}{\emph{hitXaY̯eZ}}
\newcommand{\tnif}{\emph{niXYaZ}}
\newcommand\dgs[1]{\textsubarch{#1}}
\newcommand\del[1]{\sout{#1}}

\makeatletter %Allow superscript ^ and subscript _
\catcode`_=\active%
\gdef_#1{\ensuremath{{}\sb{#1}}}%
\catcode`^=\active%
\gdef^#1{\ensuremath{{}\sp{#1}}}%
\makeatother

\begin{document}
% \pagenumbering{gobble}
% \singlespacing


\hfill \emph{Théories linguistiques 2024, Utrecht}

\section{Ordre des affixes}

    \subsection{Entrée en matière: les auxiliaires}
Dans quel ordre apparaissent les auxiliaires en français?

\pex
    \a Elle va (certainement) avoir eu gagn-é la course.
    \a \ljudge{*} Elle a eu (certainement) aller gagn-er la course.
\xe
\pex
    \a Le gâteau va (déjà) avoir été (complètement)  mangé.
    \a \ljudge{*} Le gâteau a eté avoir (déjà) aller être (complètement) mangé.
\xe

\midline{}

\bigskip
Comment s'imbriquent les éléments entre eux:
\begin{answer}{
 \begin{itemize} \tightlist
     \item Chaque élément a une influence sémantique sur le suivant. \\
    En termes techniques: chaque élément a le terme suivant \textbf{dans sa portée sémantique.} 
     \item Futur du parfait d'un passif d'un événement manger-le-gâteau.
     \item Les éléments ont une influence morphologique sur le suivant.\\
     En termes techniques: les éléments ont le terme suivant \textbf{dans leur portée morphologique} = conditionnement morphologique
\emph{avoir}_{\gsc{parfait}} déclenche la morphologie participiale sur \textit{être} (\textit{été}), \emph{être}_{\gsc{passif}} déclenche la morphologie participiale sur \textit{manger} (\emph{mangé}), etc.
 \end{itemize}
}
\end{answer}
Représentons cela formellement:

\newpage
\begin{answer}{
\ex
 \Tree
 [.TP
     [.\emph{Le gâteau} ]
     [.
         [.T\\\emph{va} ]
         [.PerfP/AspP
             [.Perf\\\emph{avoir} ]
                 [.PassP
                     [.Pass\\\emph{été} ]
                     [.VP\\\emph{mangé} ]
                 ]
         ]
     ]
 ]
\xe
}
\end{answer}


Quel est l'ordre observé en latin \citep{embick10,kastnerzu17,kastner18nllt}?
\pex
    \a \begingl
        \gla am-\=a-ve-ra-m//
        \glb \root{AIM}-\gsc{THÈME}-Perf-Passé-1\gsc{SG}//
        \glft `J'ai eu aimé'//
        \endgl
    \a \begingl
        \gla am-\=a-ve-r-\=o//
        \glb \root{AIM}-\gsc{THÈME}-Perf-Fut-1SG//
        \glft `Je vais avoir aimé'//
        \endgl
\xe
En quoi cela est-il différent du français?
\begin{answer}{
 V < Th < Perf < T.
 \begin{itemize} \tightlist
     \item $\approx$ T > Perf > V.
     \item C'est comme en français, mais dans la syntaxe en français et dans la morphologie en latin.  except in morphology rather than syntax.
     \item Il y a aussi du conditionnement morphologique (allomorphie). 
     %Still get morphological conditioning, but we'll return to that later (allomorphy).
 \end{itemize}
 }
 \end{answer}

\longline{}

    \subsection{Le Principe du Miroir}
        \subsubsection{Introduction}

Supposons ceci:

\begin{itemize} 
\item 'faire+inf.' ou `se faire+inf.' exprime le trait CAUS (=cause). Voir \textit{faire chanter}, \textit{se faire conduire}, etc.
\item  `l'un l'autre' ou `les uns les autres' exprime le trait RECIP (=reciprocité). Voir \textit{se voir l'un l'autre}, \textit{s'aider les uns les autres}, etc.
\end{itemize} 

        Comparez le sens de ces deux phrases en français:
        \pex<bouc>
       \a Les boucs, les chasseurs les ont fait se frapper les uns les autres.
        \a Les chasseurs se sont  les uns les autres fait frapper les boucs
       \xe 

Observez la syntaxe (l'ordre des mots) dans ces deux phrases. Dans quel ordre  apparaissent le verbe causatif et l'expression réciproque?  Est-ce que cela reflète la portée sémantique?

\begin{answer}{
\begin{itemize}
\item L'ordre syntaxique reflète la portée sémantique:
    \item En (\getref{bouc}a): CAUS > RECIP
    \item En (\getref{bouc}b): RECIP > CAUS
\end{itemize}}
\end{answer}


       
Dans quel ordre apparaissent les suffixes réciproque et causatif pour les contreparties de ces phrases français en chiche\^{w}a \citep{alsina99}?
\ex[exno=3] \begingl
    \gla Al\=enje a-na-mény-\textbf{án}-\uline{its}-á mb\^{u}zi//
    \glb 2.chasseurs 2\gsc{s}-\gsc{PASSÉ}-frapper-\gsc{RECIP}-\gsc{CAUS}-\gsc{V-fin} 10.boucs//
    \glft `Les boucs, les chasseurs les ont fait se frapper les uns les autres.'//
    \endgl
\xe
\ex[exno=4] \begingl
    \gla Al\=enje a-na-mény-\uline{éts}-\textbf{an}-a mb\^{u}zi//
    \glb 2.chasseurs \gsc{2S}-\gsc{PASSÉ}-frapper-\gsc{CAUS}-\gsc{RECIP}-\gsc{FV} 10.boucs//
    \glft `Les chasseurs se sont fait les uns les autres frapper les boucs.'//
    \endgl
\xe


\begin{answer}{
\begin{itemize} 
\item L'ordre observé est l'ordre nécessaire pour avoir la portée sémantique dont on a besoin pour le sens visé:
\item L'élément le moins enchâssé morphologiquement (le plus extérieur) a une portée sémantique sur les éléments le plus enchâssé.
%\item On part du verbe et on continue `vers l'extérieur': les morphèmes `ont les yeux tournés vers l'extérieur'  (\textit{outward-looking}).
\end{itemize}}
\end{answer}

\longline{}

Comparez maintenant les deux phrases suivantes en français et exprimer la relation de portée sémantique entre CAUS et RECIP:

\pex<poulie> 
\a Marie les a fait s'attacher l'un l'autre à l'arbre.
\a Ils se sont l'un l'autre fait attacher à l'arbre.
\xe 

\begin{answer}{
\begin{itemize}
\item L'ordre syntaxique reflète la portée sémantique:
    \item En (\getref{poulie}a): CAUS > RECIP
    \item En (\getref{poulie}b): RECIP > CAUS
\end{itemize}}
\end{answer}



Comparez maintenant une paire similaire en chichewa \citep{hymanmchombo92}:
\pex
    \a mang-\textbf{an}-\uline{its}\\
    attacher-\gsc{RECIP}-\gsc{CAUS}\\
    `faire s'attacher l'un l'autre'
    \a mang-\uline{its}-\textbf{an}\\
    attacher-\gsc{CAUS}-\gsc{RECIP}\\
    `se faire l'un l'autre attacher'
\xe

Comment la portée sémantique est refletée dans la syntaxe en français et dans la morphologie en chichewa'?

\begin{answer}{
    \begin{itemize}
        \item Français: l'élément qui apparaît d'abord a portée sémantique sur un élément qui apparaît après.
        \item Chichewa', le morphème qui est `plus à l'extérieur' a portée sémantique sur un élément `plus à l'intérieur'.
    \end{itemize}}
\end{answer}

À quoi ressemblerait cette langue  (\emph{chichewa'}) si elle avait des préfixes plutôt que des suffixes, et pourquoi? 
\begin{answer}{
\pex
    \a \midline{} its-an-mang
    \a \midline{}  an-its-mang
\xe
Parce que ce qui compte, c'est la hiérarchie (l'ordre des éléments les uns par rapport aux autres) plutôt que l'ordre linéaire. 
}
\end{answer}

\bigskip

Un autre exemple: le bemba tel que décrit par \cite{baker85}, qui cite \cite{givon76}.
\pex[exno=49]
    \a \begingl
        \gla Naa-mon-an-ya Mwape na Mutumba//
        \glb 1s.\gsc{S}-\gsc{passé}-voir-\gsc{recip-caus} Mwape et Mutumba//
        \glft `Mwape et Mutumba, je les ai fait se voir l'un l'autre.'//
        \endgl
    \a \begingl
        \gla Mwape na Chilufya baa-mon-eshy-ana Mutumba//
        \glb Mwape et Chilufya 3p.\gsc{S}-voir-\gsc{caus-recip} Mutumba//
        \glft `Mwape et Chilufya se sont  l'un l'autre fait voir  Mutumba.'//
        \endgl
\xe

        \subsubsection{Ailleurs qu'en bantou}


        Comparez la portée respective du passé, du verbe \textit{dire} et du marqueur épistémique (\textit{probablement}) dans les  phrases suivantes:

        \pex<neige>
        \a Il a dit que, probablement, il partait  demain.
        \a Probablement qu'il a dit qu'il partait demain.
        \xe 
        
\begin{answer}{
\begin{itemize}
    \item En (\getref{neige}a): PASSE > \textit{dire} > \textit{probablement}
    \item En (\getref{neige}b): \textit{probablement} > PASSE > \textit{dire}
\end{itemize}
}
\end{answer}
Comparez maintenant les phrases du français avec les contreparties de ces phrases en Yupik \citep[43]{mithun99}:
\pex
    \a \begingl
        \gla ayag-ciq-\textbf{yugnarqe}-\uline{ni}-llru-u-q//
        \glb aller-\gsc{FUT}-probablement-dire-\gsc{PASSÉ-INDIC.INTR}-3\gsc{SG}//
        \glft `Il a dit  qu'il allait probablement partir.'//
    \endgl
    \a \begingl
        \gla ayag-ciq-\uline{ni}-llru-\textbf{yugnarqe}-u-q//
        \glb aller-\gsc{FUT}-dire-\gsc{PASSE}-probablement-\gsc{INDIC.INTR}-3\gsc{SG}//
        \glft `Probablement qu'il a dit qu'il allait partir.'//
    \endgl
\xe



% \ex \begingl
%     \gla ayag-yug-umi-ite-qapiar-tu-a//
%     \glb go-want-be.in.state-not-really-\gsc{INDIC.INTR}-1\gsc{SG}//
%     \glft `I really don't want to go.'//
%     \endgl
% \xe
% (Why do we gloss these sometimes as words and sometimes as affixes? Because.)

Comparez maintenant les phrases suivantes en français:

\pex<stop>
\a  Il a arrêté de neiger dans la nuit (il ne neige plus pendant la nuit)
\a  Il a arrêté de neiger dans la nuit (il a neigé toute la journée).
\xe 

Laquelle des deux phrases (\getref{stop}a/b) a le même sens que la phrase (\getref{stop2}) ci-dessous, et pourquoi? (Réfléchissez à la portée relative de \textit{arrêter} et \textit{dans la nuit})

\pex<stop2> 
Dans la nuit il a arrêté de neiger.
\xe 

\begin{answer}{
\begin{itemize}
    \item C'est (\getref{stop}b) qui a le même sens que (\getref{stop2}), parce que l'ordre syntaxique de (\getref{stop2}) reflète de manière transparente la portée sémantique en (\getref{stop}b): \textit{dans la nuit} > \textit{arrêter}
    \item Le fait qu'une même phrase peut exprimer une portée sémantique différente montre déjà que le tableau jusqu'ici est simplifié: parfois, la hiérarchie syntaxique ne reflète pas la hiérarchie sémantique.
    \item Nous reviendrons sur ce point  plus tard.
\end{itemize}
}    
\end{answer}

Appariez maintenant les phrases françaises avec les phrases ci-dessous en
Oji-Cree \citep{slavin05uot} dans \cite{rice11}:
\pex
    \a \begingl
        \gla nipaa-ishkwaa-sookihpwan//
        \glb dans.la.nuit-arrêter-être.neigeant//
        \glft \longline{} // % `It stopped snowing at night.' (was snowing the whole day)//
    \endgl
    \a \begingl
        \gla ishkwaa-niipaa-sookihpawn//
        \glb arrêter-dans.la.nuit-être.neigeant//
        \glft \longline{} // %`It stopped snowing at night.' (does not snow at night anymore)//
    \endgl
\xe

Même exercice avec les deux phrases français ci-dessous (réfléchissez à la portée relative de \textit{à l'insu de tous} (= \textit{secrètement}) et \textit{très vite}):

\pex<insu> 
\a A l'insu de tous, il mange très vite (personne ne sait qu'il mange très vite)
\a Très vite, il mange à l'insu de tous
\xe 

Quelle phrase française correspond à quelle phrase en Oji-Cree et pourquoi?

\pex<11'>
    \a \begingl
        \gla kiimooci-kishahtapi-wiihsini//
        \glb à.l'insu.de.tous-vite-manger//
        \glft \longline{} // % `He secretly eats fast.' (nobody knows that he eats fast)//
    \endgl
    \a \begingl
        \gla kishahtapi-kiimooci-wiihsini//
        \glb vite-à.l'insu.de.tous-manger//
        \glft \longline{} // % `He eats secretly (nobody knows that he eats) and he does it fast.'//
    \endgl
\xe

\begin{answer}{
\begin{itemize}
    \item (\getref{insu}a) exprime le même sens que (\getref{11'}a).
   \item  (\getref{insu}b) exprime le même sens que (\getref{11'}b).
   \item L'emboîtement morphologique ou syntaxique reflète l'emboîtement sémantique.
\end{itemize}
}\end{answer}


    \subsection*{Résumé}
\begin{itemize} \tightlist
    \item Certaines catégories suivent un ordre rigide, comme par exemple les auxiliaires.
    \item D'autres catégories suivent un ordre variable qui dépend de la \textit{portée} sémantique.
    \item La portée morphologique reflète systématiquement la portée sémantique.
\end{itemize}

Principe du Miroir:
\begin{itemize}
\item L'ordre des morphèmes \textit{X}...\textit{Y} (en morphologie) est en relation systématique avec l'ordre des constituants \textit{XP}...\textit{YP} (en syntaxe).
    \item L'ordre relatif des affixes peut être prédit à partir de propriétés sémantiques.
\end{itemize}

Références clé:
\cite{baker85} a trouvé la formulation du principle du miroir, mais  \cite{muysken81,muysken88} est le premier à avoir trouvé la généralisation. 

  

\newpage
\bibliographystyle{linquiry2}
\bibliography{lingxbib}

\end{document}

\begin{comment}

        \subsubsection{Passivation}
En chichewa', l'applicatif peut être utilisé pour les instruments \citep{alsina99}:
\pex[exno=10] 
    \a \begingl
        \gla Ms\=odzi a-na-dúl-\textbf{ír}-a \textbf{nkhw\^{a}ngwa} uk\=onde//
        \glb 1.fisherman \gsc{1S}-\gsc{PASSÉ}-couper-\gsc{APPL-VF} 9.hache 14.filet//
        \glft `Le pêcheur a coupé le filet avec une hache.'//
        \endgl
\xe

Essayons de passiver de cette manière: `La hache a été utilisée pour couper le filet (par le pêcheur). L'instrument va devenir le sujet. Quel va être l'ordre de  \gsc{Appl} et \gsc{Pass}?

\pagebreak 
\pex[exno=10]
    \a[label=b] \begingl
        \gla Nkhw\^{a}ngwa i-na-dúl-\textbf{ír}-\uline{idw}-á úk\=onde (ndí ms\=odzi)//
        \glb 9.hache 9\gsc{S}-\gsc{PASSÉ}-couper-\gsc{APPL}-\gsc{PASS-VF} 14.filet par 1.pêcheur//
        \glft `La hache a été utilisée pour couper le filet (par le pêcheur).'//
        \endgl
\xe

Schématiquement, en ignorant les arguments:\\
\Tree
[.PassP
    [.Pass ]
    [.ApplP
        [.Appl ]
        \qroof{\emph{couper}}.vP
    ]
]    


Le chi-Mwi:ni dans \cite{baker85}, données de \cite{kisseberthabasheikh77}:
\pex[exno=56]   
    \a \begingl
        \gla Nu:ru Ø-chi-tes-ete chibu:ku//
        \glb Nuru \gsc{S-O}-apporter-\gsc{asp} livre//
        \glft `Nuru a apporté le livre.'//
        \endgl
    \a \begingl
        \gla Nu:ru ø-m-tet-\textbf{el}-ele mwa:limu chibu:ku//
        \glb Nuru \gsc{S-O}-apporter-\gsc{appl-asp} professeur livre//
        \glft `Nuru a apporté le livre au professeur.'//
        \endgl
    \a \begingl
        \gla Mwa:limu ø-tet-\textbf{el}-el-\uline{a} chibu:ku na Nu:ru//
        \glb professeur \gsc{S}-apporter-\gsc{appl-asp-pass} livre par Nuru//
        \glft Littéralement `Le professeur a été apporté le livre par Nuru (en bon français: le livre a été apporté au professeur par Nuru).'//
        \endgl
\xe

Le kinyarwanda dans \cite{baker85}, données de  \cite{kimenyi80}:
\pex[exno=57]
    \a \begingl
        \gla Umugabo a-ra-andik-a ibaruwa \textbf{n'i}-ikaramu//
        \glb homme \gsc{S-prés}-écrire-\gsc{asp} lettre avec-stylo//
        \glft `L'homme a écrit  [sic] la lettre avec le stylo.'//
        \endgl
    \a \begingl
        \gla Umugabo a-ra-andik-\textbf{iish}-a ibaruwa ikaramu//
        \glb homme \gsc{S-prés}-écrire-\gsc{instr-asp} lettre stylo//
        \glft `L'homme a écrit  [sic] la lettre avec le stylo.'//
        \endgl
    \a \begingl
        \gla Ikaramu i-ra-andik-\textbf{iish}-\uline{w}-a ibaruwa n'umugabo//
        \glb stylo \gsc{S-prés}-écrire-\gsc{instr-pass-asp} lettre par-homme//
        \glft `Le stylo a été écrit-avec  [sic] la lettre par l'homme.'//
        \endgl
    \a \begingl
        \gla Ibaruwa i-ra-andik-\textbf{iish}-\uline{w}-a ikaramu n'umugabo//
        \glb lettre \gsc{S-prés}-écrire-\gsc{instr-pass-asp} stylo par-homme//
        \glft `La lettre a été écrite avec le stylo par l'homme.'//
        \endgl
\xe
\end{comment}