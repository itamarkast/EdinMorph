\documentclass[a4paper,12pt]{article}
%\usepackage[utf8]{inputenc}
\usepackage {fontspec} %,xunicode} %,xltxtra}
%	\setmainfont{Times New Roman}
\setmainfont[
Ligatures={TeX,Common},
PunctuationSpace=0,
Numbers={Proportional},
BoldFont = LibertinusSerif-Semibold.otf,
ItalicFont = LibertinusSerif-Italic.otf,
BoldItalicFont = LibertinusSerif-SemiboldItalic.otf,
BoldSlantedFont = LibertinusSerif-Semibold.otf,
SlantedFont    = LibertinusSerif-Regular.otf,
SlantedFeatures = {FakeSlant=0.25},
BoldSlantedFeatures = {FakeSlant=0.25},
SmallCapsFeatures = {FakeSlant=0},
]{LibertinusSerif-Regular.otf}

\usepackage[margin={1in}]{geometry}

\usepackage[dvipsnames]{xcolor}
\usepackage[linkcolor=purple,citecolor=ForestGreen,colorlinks=true,urlcolor=purple,pagebackref=false]{hyperref}

\usepackage{natbib}
\bibpunct[:]{(}{)}{;}{a}{}{,}
\setlength{\bibsep}{0pt plus 0.3ex}
\usepackage{graphicx}
\usepackage{multicol}
\usepackage{amsmath}
\usepackage{expex} %Linguistics examples
\lingset{aboveexskip=0.2ex,belowexskip=0.8ex,aboveglftskip=-0.2ex,interpartskip=0.2ex,labelwidth=!6pt,belowpreambleskip=0.1ex} %, *=*?}

\usepackage{mdwlist}  %
\usepackage[normalem]{ulem}
\usepackage{array} % >{\...} in tabulars
\setlength\parindent{0pt}
\usepackage{dashrule} %\hdashrule

\usepackage{lastpage}
\usepackage{fancyhdr}
\addtolength{\headheight}{2.5pt}
\pagestyle{fancy} %or fancyplain
\lhead{Morphology 2023-24 (UoE)}
\rhead{Jigsaw group A}
\fancyfoot[C]{\thepage~of \pageref{LastPage}}
\fancypagestyle{plain}{%
\fancyhf{} % clear all header and footer fields
\fancyfoot[C]{Page \thepage~of \pageref{LastPage}} % except the center
\renewcommand{\headrulewidth}{0pt}
\renewcommand{\footrulewidth}{0pt}}

\definecolor{edir}{RGB}{193,0,67}
\definecolor{edib}{RGB}{0,50,95}
\newcommand\ruler{{\centering \color{edir} \rule[-0.5em]{0.6\columnwidth}{0.4pt}\par}}

\renewcommand\root[1]{$\sqrt{\text{#1}}$}

\usepackage{pifont}
\newcommand{\cmark}{\ding{51}}% 52
\newcommand{\xmark}{\ding{55}}%
\newcommand{\hand}{\ding{43}}

\newcommand\gsc[1]{\textsc{\lowercase{#1}}} %for glossing in small caps - comment out to return to caps

\makeatletter %Allow superscript ^ and subscript _
\catcode`_=\active%
\gdef_#1{\ensuremath{{}\sb{#1}}}%
\catcode`^=\active%
\gdef^#1{\ensuremath{{}\sp{#1}}}%
\makeatother

\title{Jigsaw group A}
\date{Morphology 2023-24, allomorphy}
\author{}

\begin{document}

What allomorphy is happening in these examples? Which morpheme is being \textbf{conditioned} (or \textbf{triggered}) by which element? Try and identify the same pattern that holds in all cases. % syntactic, outward-looking


\pex
\a \emph{go} (today) 
\a  \emph{went} (yesterday) 
\xe


\pex
\a \emph{good}		 
\a  \emph{better} 
\a  \emph{best} 
\xe

What would the \textbf{tree structure} for this interaction look like? Which node is looking at which?
\vspace{3cm}


Compare with a similar phenomenon for Applicatives in Georgian, here with the verb \emph{rbina} `run' (see if you can figure out the glosses for the different morphemes):
\pex 
\a mo-\textbf{m}-\uline{i}-rbina `\textbf{I} had run here'
\a mo-\textbf{gv}-\uline{i}-rbina `\textbf{we} had run here'
\a mo-\textbf{g}-\uline{i}-rbina(t) `\textbf{you} all had run here'\\
\hdashrule[2pt][c]{0.5\linewidth}{1pt}{1.5mm}
\a mo-\textbf{Ø}-\uline{u}-rbina(t) `\textbf{he/she/they} had run here'
\xe

%\ex \gsc{appl} \lra{} $\begin{cases}
	%	u & / [3~[~\trace{}~]] \\
	%	i
	%\end{cases}$
	%\xe

\bigskip

Also for suppletive verbs in Hiaki:

\ex
\begin{tabular}{l >{\em}l >{\em}l l}
	& \gsc{SG} & \gsc{PL} & \\\hline
	a.& vuite & tenne & `run' \\
	b.& siika & saka & `go'\\
	c.& weama & rehte & `wander'\\
	d.& kivake & kiime & `enter'\\
	e.& vo'e & to'e & `lie'\\
	f.& weye & kaate & `walk'\\
	g.& mea & sua & `kill' (\gsc{sg}$\sim$\gsc{pl} object)\\
\end{tabular}
\xe

\vfill

If you finish with time to spare: what other interactions would you expect to exist?


\end{document}