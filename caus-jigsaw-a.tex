\documentclass[a4paper,12pt]{article}
%\usepackage[utf8]{inputenc}
\usepackage {fontspec} %,xunicode} %,xltxtra}
%	\setmainfont{Times New Roman}
\setmainfont[
Ligatures={TeX,Common},
PunctuationSpace=0,
Numbers={Proportional},
BoldFont = LibertinusSerif-Semibold.otf,
ItalicFont = LibertinusSerif-Italic.otf,
BoldItalicFont = LibertinusSerif-SemiboldItalic.otf,
BoldSlantedFont = LibertinusSerif-Semibold.otf,
SlantedFont    = LibertinusSerif-Regular.otf,
SlantedFeatures = {FakeSlant=0.25},
BoldSlantedFeatures = {FakeSlant=0.25},
SmallCapsFeatures = {FakeSlant=0},
]{LibertinusSerif-Regular.otf}

\usepackage[margin={1in}]{geometry}

\usepackage[dvipsnames]{xcolor}
\usepackage[linkcolor=purple,citecolor=ForestGreen,colorlinks=true,urlcolor=purple,pagebackref=false]{hyperref}

\usepackage{natbib}
\bibpunct[:]{(}{)}{;}{a}{}{,}
\setlength{\bibsep}{0pt plus 0.3ex}
\usepackage{graphicx}
\usepackage{multicol}
\usepackage{amsmath}
\usepackage{expex} %Linguistics examples
\lingset{aboveexskip=0.2ex,belowexskip=0.8ex,aboveglftskip=-0.2ex,interpartskip=0.2ex,labelwidth=!6pt,belowpreambleskip=0.1ex} %, *=*?}

\usepackage{mdwlist}  %
\usepackage[normalem]{ulem}
\usepackage{array} % >{\...} in tabulars
\setlength\parindent{0pt}
\usepackage{dashrule} %\hdashrule

\usepackage{lastpage}
\usepackage{fancyhdr}
\addtolength{\headheight}{2.5pt}
\pagestyle{fancy} %or fancyplain
\lhead{Morphology 2023-24 (UoE)}
\rhead{Jigsaw group A}
\fancyfoot[C]{\thepage~of \pageref{LastPage}}
\fancypagestyle{plain}{%
\fancyhf{} % clear all header and footer fields
\fancyfoot[C]{Page \thepage~of \pageref{LastPage}} % except the center
\renewcommand{\headrulewidth}{0pt}
\renewcommand{\footrulewidth}{0pt}}

\definecolor{edir}{RGB}{193,0,67}
\definecolor{edib}{RGB}{0,50,95}
\newcommand\ruler{{\centering \color{edir} \rule[-0.5em]{0.6\columnwidth}{0.4pt}\par}}

\renewcommand\root[1]{$\sqrt{\text{#1}}$}

\usepackage{pifont}
\newcommand{\cmark}{\ding{51}}% 52
\newcommand{\xmark}{\ding{55}}%
\newcommand{\hand}{\ding{43}}

\newcommand\gsc[1]{\textsc{\lowercase{#1}}} %for glossing in small caps - comment out to return to caps

\makeatletter %Allow superscript ^ and subscript _
\catcode`_=\active%
\gdef_#1{\ensuremath{{}\sb{#1}}}%
\catcode`^=\active%
\gdef^#1{\ensuremath{{}\sp{#1}}}%
\makeatother

\title{Jigsaw group A} % transitive, verb-derived, cause O to be V-ed, relating causing and caused events
\date{Morphology 2023-24, causatives}
\author{}

\begin{document}

In this jigsaw puzzle we will look at different kinds of causatives. For simplicity, assume that all the forms glossed \gsc{CAUS} are the same kind of element (albeit in different languages) - the exact syntactic label is not important right now.

In this expert group, see if you can find out what is being caused, and to what/whom. Characterise this in terms of grammatical roles (subject, object, etc).

\section{English}

\pex \a Tyler opened the door.
    % \a The door opened.
    \a Tyler caused the door to open.
\xe

\pex \a Kim broke the vase.
    % \a The vase broke.
    \a Kim caused the vase to break.
\xe

\pex \a Kim taught Chris.
    % \a Chris learned.
    \a Kim caused Chris to learn.
\xe

% hang hung
% hang hanged

% rise rose
% raise raised

% rise raise


% destroy the tower
% * the tower destroyed

% They broke \{their promise / the world record}
% * \{The promise / The world record\} broke


\section{Hebrew}

Only the forms glossed with \gsc{CAUS} here are the ``causative'' ones for our purposes. The pattern here should be similar to the one in English above.

\pex
    \a \begingl
        \gla teo kara et ha-sefer//
        \glb Theo read.\gsc{PAST} \gsc{ACC} the-book//
        \glft `Theo read the book.'//
    \endgl
    \a \begingl
        \gla teo hekri et ha-sefer le-axiv//
        \glb Theo read.\gsc{PAST}-\gsc{CAUS} \gsc{ACC} the-book to-his.brother//
        \glft `Theo read the book out to his brother.'//
    \endgl
\xe
\pex
    \a \begingl
        \gla ha-maxʃev kalat et ha-ʃeder//
        \glb the-computer received.\gsc{PAST} \gsc{acc} the-transmission//
        \glft `The computer received the transmission.//
    \endgl
    \a \begingl
        \gla ha-maxʃev heklit et ha-toxnit//
        \glb the-computer received.\gsc{PAST}-\gsc{CAUS} \gsc{ACC} the-program//
        \glft `The computer recorded the program.'//
    \endgl
\xe
\pex
    \a \begingl
        \gla ha-sefer nafal//
        \glb the-book fell.\gsc{PAST}//
        \glft `The book fell.'//
    \endgl
    \a \begingl
        \gla teo hepil et ha-sefer//
        \glb Theo fell.\gsc{PAST}-\gsc{CAUS} \gsc{ACC} the-book//
        \glft `Theo dropped the book.'//
    \endgl
\xe

\newpage
\section{Quechua}

And the same pattern for \gsc{CAUS}:

\pex
    \a \begingl
    \gla Yaku t'impu-rqa//
    \glb Water boil-\gsc{PAST}//
    \glft `The water boiled.'//
    \endgl
    
    \a \begingl
    \gla Wayna yaku-ta t'impu-chi-rqa//
    \glb boy water-\gsc{ACC} boil-\gsc{CAUS}-\gsc{PAST}//
    \glft `The boy boiled the water.'//
    \endgl
\xe

\ex \begingl
    \gla Juan Marya-ta phiña-chi-rqa//
    \glb John Mary-\gsc{ACC} anger-\gsc{CAUS}-\gsc{PAST}//
    \glft `John angered Mary.'//
    \endgl
    % \a \begingl
    % \gla Marya phiña-\uline{ku}-rqa//
    % \glb Mary anger-REFL-\gsc{PAST}//
    % \glft `Mary got angry.'//
    % \endgl
\xe

\section{Romance}
And the same pattern again, with `make':

\pex
    \a \begingl
        \gla \rightcomment{[French]}Marie  fit  reparer  la  voiture  (à  Jean). //
        \glb Marie made repair the car at Jean//
        \glft `Marie got the car fixed (by Jean)', `Marie made John fix the car.'//
    \endgl
    \a \begingl
        \gla \rightcomment{[Italian]}Maria  fece  riparare  la  macchina  (a  Gianni).  //
        \glb Maria made repair the car at Gianni//
        \glft `Maria got the car fixed (by Gianni).'//
    \endgl
    \a \begingl
        \gla \rightcomment{[Spanish]}Maria  hizo  arreglar  el  coche  (a  Juan). //
        \glb Maria made repair the car at Juan//
        \glft `Maria got the car fixed (by Juan).'//
    \endgl
\xe

% \section{Hiaki}

% \ex \begingl
%     \gla Juan vaso-ta ham-ta-tevo-k//
%     \glb Juan glass-\gsc{ACC} break-\gsc{TR-CAUS-PRF}//
%     \glft `Juan had the glass broken.'//
%     \endgl
% \xe


\section{Analysis}

If you have time: what would a formal analysis look like? Specifically, what would the relationship be between the Causer, the Causee, the verb/root, and the element \gsc{CAUS}?

\end{document}