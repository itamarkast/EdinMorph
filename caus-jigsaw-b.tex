\documentclass[a4paper,12pt]{article}
%\usepackage[utf8]{inputenc}
\usepackage {fontspec} %,xunicode} %,xltxtra}
%	\setmainfont{Times New Roman}
\setmainfont[
Ligatures={TeX,Common},
PunctuationSpace=0,
Numbers={Proportional},
BoldFont = LibertinusSerif-Semibold.otf,
ItalicFont = LibertinusSerif-Italic.otf,
BoldItalicFont = LibertinusSerif-SemiboldItalic.otf,
BoldSlantedFont = LibertinusSerif-Semibold.otf,
SlantedFont    = LibertinusSerif-Regular.otf,
SlantedFeatures = {FakeSlant=0.25},
BoldSlantedFeatures = {FakeSlant=0.25},
SmallCapsFeatures = {FakeSlant=0},
]{LibertinusSerif-Regular.otf}

\usepackage[margin={1in}]{geometry}

\usepackage[dvipsnames]{xcolor}
\usepackage[linkcolor=purple,citecolor=ForestGreen,colorlinks=true,urlcolor=purple,pagebackref=false]{hyperref}

\usepackage{natbib}
\bibpunct[:]{(}{)}{;}{a}{}{,}
\setlength{\bibsep}{0pt plus 0.3ex}
\usepackage{graphicx}
\usepackage{multicol}
\usepackage{amsmath}
\usepackage{expex} %Linguistics examples
\lingset{aboveexskip=0.2ex,belowexskip=0.8ex,aboveglftskip=-0.2ex,interpartskip=0.2ex,labelwidth=!6pt,belowpreambleskip=0.1ex} %, *=*?}

\usepackage{mdwlist}  %
\usepackage[normalem]{ulem}
\usepackage{array} % >{\...} in tabulars
\setlength\parindent{0pt}
\usepackage{dashrule} %\hdashrule

\usepackage{lastpage}
\usepackage{fancyhdr}
\addtolength{\headheight}{2.5pt}
\pagestyle{fancy} %or fancyplain
\lhead{Morphology 2023-24 (UoE)}
\rhead{Jigsaw group B}
\fancyfoot[C]{\thepage~of \pageref{LastPage}}
\fancypagestyle{plain}{%
\fancyhf{} % clear all header and footer fields
\fancyfoot[C]{Page \thepage~of \pageref{LastPage}} % except the center
\renewcommand{\headrulewidth}{0pt}
\renewcommand{\footrulewidth}{0pt}}

\definecolor{edir}{RGB}{193,0,67}
\definecolor{edib}{RGB}{0,50,95}
\newcommand\ruler{{\centering \color{edir} \rule[-0.5em]{0.6\columnwidth}{0.4pt}\par}}

\renewcommand\root[1]{$\sqrt{\text{#1}}$}

\usepackage{pifont}
\newcommand{\cmark}{\ding{51}}% 52
\newcommand{\xmark}{\ding{55}}%
\newcommand{\hand}{\ding{43}}

\newcommand\gsc[1]{\textsc{\lowercase{#1}}} %for glossing in small caps - comment out to return to caps

\makeatletter %Allow superscript ^ and subscript _
\catcode`_=\active%
\gdef_#1{\ensuremath{{}\sb{#1}}}%
\catcode`^=\active%
\gdef^#1{\ensuremath{{}\sp{#1}}}%
\makeatother

\title{Jigsaw group B} % complex events, VoiceP-derived, cause O to V
\date{Morphology 2023-24, causatives}
\author{}

\begin{document}

In this jigsaw puzzle we will look at different kinds of causatives. For simplicity, assume that all the forms glossed \gsc{CAUS} are the same kind of element (albeit in different languages) - the exact syntactic label is not important right now.

In this expert group, see if you can find out what is being caused, and to what/whom. Characterise this in terms of grammatical roles (subject, object, etc).
% \section{English}


\section{Hebrew}

Only the forms glossed with \gsc{CAUS} here are the ``causative'' ones for our purposes.

\pex
    \a \begingl
        \gla adam axal et ha-xumus//
        \glb Adam ate-\gsc{PAST} \gsc{ACC} the-hummus//
        \glft `Adam ate the hummus.'//
    \endgl
    \a \begingl
        \gla adam heexil et teo xumus//
        \glb Adam ate.\gsc{CAUS-\gsc{PAST}} \gsc{ACC} Theo hummus//
        \glft `Adam fed Theo hummus.'//
    \endgl
\xe

\pex
    \a \begingl
        \gla teo lavaʃ et ha-meil//
        \glb Theo wore-\gsc{PAST} \gsc{ACC} the-coat//
        \glft `Theo wore the coat.'//
    \endgl
    \a \begingl
        \gla teo helbiʃ et axiv be-meil//
        \glb Theo wore.\gsc{CAUS-\gsc{PAST}} \gsc{ACC} his.brother in-coat//
        \glft `Theo dressed his brother in a coat.'//
    \endgl
\xe
\pex
    \a \begingl
        \gla teo xatam al ha-mismax//
        \glb Theo signed-\gsc{PAST} on the-document//
        \glft `Theo signed the document.'//
    \endgl

    \a \begingl
        \gla teo hextim et ha-menahelet al ha-mismax//
        \glb Theo signed.\gsc{CAUS-\gsc{PAST}} \gsc{ACC} the-manager on the-document//
        \glft `Theo signed the manager on the document.'//
    \endgl
\xe


\section{Japanese}

The same pattern can be found in Japanese:

\ex \begingl
    \gla Taroo-ga Hanako-o ik-ase-ta//
    \glb Taro-\gsc{NOM} Hanako-\gsc{ACC} go-\gsc{CAUS}-\gsc{PAST}//
    \glft `Taro made Hanako go.'//
    \endgl
\xe

\ex \begingl
    \gla Kotosi-wa dekinai gakusei-o hue-sase-ta//
    \glb This.year-\gsc{Topic} poor students-\gsc{ACC} increase-\gsc{CAUS}-\gsc{PAST}//
    \glft `This year, we caused (the number of) poor students to increase.'//
    \endgl
\xe

% kow-as-ase
% break-\gsc{CAUS}-\gsc{CAUS}
% ‘make someone break something’

% ugok-as-ase
%  move-\gsc{CAUS}-\gsc{CAUS}
% ‘make someone move something

\ex \begingl
    \gla Taroo-wa Hanako-ni hanasi-o tutae-sase-ta//
    \glb Taro-\gsc{Topic} Hanako-\gsc{DAT} story-\gsc{ACC} convey-\gsc{CAUS}-\gsc{PAST}//
    \glft `Taro made Hanako convey a story.'//
    \endgl
\xe

\newpage
\section{Quechua}
And the same pattern in Quechua:

\ex \begingl
    \gla Marya Juan-man libru-ta rikhu-chi-rqa//
    \glb Mary Juan-\gsc{DAT} book-\gsc{ACC} see-\gsc{CAUS}-\gsc{PAST}//
    \glft `Mary showed John the book.'//
\endgl
\xe

\ex \begingl
    \gla Juan puñu-n//
    \glb Juan sleep-\gsc{3SUBJ}//
    \glft `Juan sleeps.'//
    \endgl
\xe
\ex \ljudge{*} \begingl
    \gla Maria Juan-ta puñu-n//
    \glb Maria Juan-\gsc{ACC} sleep-\gsc{3SUBJ}//
    \glft (int.~‘Maria sleeps Juan.’)//
    \endgl
\xe
\ex \begingl
    \gla Maria Juan-ta puñu-chi-n//
    \glb Maria Juan-\gsc{ACC} sleep-\gsc{CAUS}-\gsc{3SUBJ}//
    \glft ‘Maria makes Juan sleep.’//
    \endgl
\xe

% And when the embedded verb is transitive:

\pex \a \begingl
    \gla Alqo wayna-ta khani-rqa//
    \glb Dog boy-\gsc{ACC} bite-\gsc{PAST}//
    \glft ‘The dog bit the boy.’//
    \endgl
    
    \a \begingl
    \gla Juan alqo-wan wayna-ta khani-chi-rqa.//
    \glb Juan dog-\gsc{INSTR} boy-\gsc{ACC} bite-\gsc{CAUS}-\gsc{PAST}//
    \glft `Juan made the dog bite the boy.' (literally `Juan made-bite the boy using the dog')//
    \endgl
\xe


% \section{Hiaki}
% \ex \begingl
%     \gla Juan Maria-ta vaso-ta ham-ta-tua-k//
%     \glb Juan Maria-\gsc{ACC} glass-\gsc{ACC} break-\gsc{TR}-\gsc{CAUS}-\gsc{PERF}//
%     \glft `Juan made Maria break the glass.'//
%     \endgl
% \xe

\section{Turkish}
And also in Turkish:

\ex \begingl
  \gla ben kIz-a maymun-u sat-tIr-dI-m//
  \glb I girl-\gsc{DAT} monkey-\gsc{ACC} sell-\gsc{CAUS}-\gsc{PAST}-1SG//
  \glft `I made the girl sell the monkey'//
  \endgl
\xe 

\ex \begingl
  \gla kIz-I maymun-a bak-tIr-dI-m//
  \glb girl-\gsc{ACC} monkey-\gsc{DAT} look-\gsc{CAUS}-\gsc{PAST}-1SG//
  \glft `I made/had/let the girl look at the monkey'//
  \endgl
\xe 


\ex \begingl
  \gla kIz-I maymun-dan kork-tur-uyor-um//
  \glb girl-\gsc{ACC} monkey-ABL fear-\gsc{CAUS}-PRES-1SG//
  \glft `I am making the girl fear the monkey'//
  \endgl
\xe 

% \section{IsiXhosa}
% \pex 
%     \a Intengiso i-fun-is-e inkwenkwe ithoyi.
% 9advertisement 9SBJ-want-\gsc{CAUS}-PRF 9boy 9toy
% ‘The advertisement made the boy want a toy.’

%     \a uDallas_i w-aphul-is-e uZoli_j iglasi ngabom{i/*j}.
% 1Dallas 1SBJ-break.TR-\gsc{CAUS}-PRF 1Zoli 9glass on.purpose
% ‘Dallas [[made Zoli break the glass] on purpose].’
% *‘Dallas [made [Zoli break the glass on purpose]].’
% \xe


% (3) Intengiso y-enz-e ukuba inkwenkwe i-fun-e ithoyi.
% 9advertisement 9SBJ-make-PRF COMP 9boy 9SBJ-want-SBJV 9toy
% ‘The advertisement made the boy want a toy.’


\section{Analysis}

What would a formal analysis look like? Specifically, what would the relationship be between the Causer, the Causee, the verb/root, and the element \gsc{CAUS}?




\end{document}