\documentclass[a4paper,12pt]{article}
%\usepackage[utf8]{inputenc}
\usepackage {fontspec} %,xunicode} %,xltxtra}
%	\setmainfont{Times New Roman}
\setmainfont[
Ligatures={TeX,Common},
PunctuationSpace=0,
Numbers={Proportional},
BoldFont = LibertinusSerif-Semibold.otf,
ItalicFont = LibertinusSerif-Italic.otf,
BoldItalicFont = LibertinusSerif-SemiboldItalic.otf,
BoldSlantedFont = LibertinusSerif-Semibold.otf,
SlantedFont    = LibertinusSerif-Regular.otf,
SlantedFeatures = {FakeSlant=0.25},
BoldSlantedFeatures = {FakeSlant=0.25},
SmallCapsFeatures = {FakeSlant=0},
]{LibertinusSerif-Regular.otf}

\usepackage[margin={1in}]{geometry}

\usepackage[dvipsnames]{xcolor}
\usepackage[linkcolor=purple,citecolor=ForestGreen,colorlinks=true,urlcolor=purple,pagebackref=false]{hyperref}

\usepackage{natbib}
\bibpunct[:]{(}{)}{;}{a}{}{,}
\setlength{\bibsep}{0pt plus 0.3ex}
\usepackage{graphicx}
\usepackage{multicol}
\usepackage{amsmath}
\usepackage{expex} %Linguistics examples
\lingset{aboveexskip=0.2ex,belowexskip=0.8ex,aboveglftskip=-0.2ex,interpartskip=0.2ex,labelwidth=!6pt,belowpreambleskip=0.1ex} %, *=*?}

\usepackage{mdwlist}  %
\usepackage[normalem]{ulem}
\usepackage{array} % >{\...} in tabulars
\setlength\parindent{0pt}
\usepackage{dashrule} %\hdashrule

\usepackage{lastpage}
\usepackage{fancyhdr}
\addtolength{\headheight}{2.5pt}
\pagestyle{fancy} %or fancyplain
\lhead{Morphology 2023-24 (UoE)}
\rhead{Jigsaw group C}
\fancyfoot[C]{\thepage~of \pageref{LastPage}}
\fancypagestyle{plain}{%
\fancyhf{} % clear all header and footer fields
\fancyfoot[C]{Page \thepage~of \pageref{LastPage}} % except the center
\renewcommand{\headrulewidth}{0pt}
\renewcommand{\footrulewidth}{0pt}}

\definecolor{edir}{RGB}{193,0,67}
\definecolor{edib}{RGB}{0,50,95}
\newcommand\ruler{{\centering \color{edir} \rule[-0.5em]{0.6\columnwidth}{0.4pt}\par}}

\renewcommand\root[1]{$\sqrt{\text{#1}}$}

\usepackage{pifont}
\newcommand{\cmark}{\ding{51}}% 52
\newcommand{\xmark}{\ding{55}}%
\newcommand{\hand}{\ding{43}}

\newcommand\gsc[1]{\textsc{\lowercase{#1}}} %for glossing in small caps - comment out to return to caps

\makeatletter %Allow superscript ^ and subscript _
\catcode`_=\active%
\gdef_#1{\ensuremath{{}\sb{#1}}}%
\catcode`^=\active%
\gdef^#1{\ensuremath{{}\sp{#1}}}%
\makeatother

\title{Jigsaw group C} % lexical, unpredictable transitive meaning, root-derived
\date{Morphology 2023-24, causatives}
\author{}

\begin{document}

In this jigsaw puzzle we will look at different kinds of causatives. For simplicity, assume that all the forms glossed \gsc{CAUS} are the same kind of element (albeit in different languages) - the exact syntactic label is not important right now.

In this expert group, we will try to see how \gsc{CAUS} can interact with the verb.

\section{Japanese}

What would be the literal meaning of each of these examples, and what is the actual meaning?

\ex 
    \begingl
        \gla Taroo-ga zisyoku-o niow-ase-ta//
        \glb Taro-\gsc{NOM} resignation-\gsc{ACC} smell-\gsc{CAUS}-\gsc{PAST}//
        \glft `Taro hinted at resigation.'//
    \endgl
\xe
\ex
    \begingl
        \gla mimi-o sum-ase//
        \glb ear-\gsc{ACC} clear-\gsc{CAUS}//
        \glft `listen carefully'//
    \endgl
\xe
\ex
    \begingl
        \gla hana-o sak-ase//
        \glb flower-\gsc{ACC} bloom-\gsc{CAUS}//
        \glft `engage in heatedly'//
    \endgl
\xe

\section{German}

The prefix \emph{be}- creates causative verbs in German:
\pex
    \a \emph{enden} `to end (intransitive)'
    \a \emph{beenden} `to end something (transitive)'
\xe
\pex
    \a \emph{malen} `to draw, paint'
    \a \emph{bemalen} `to paint a surface (completely)'
\xe
\pex
    \a \emph{Nachricht} `a message'
    \a \emph{benachrichtigen} `to inform'
\xe

% The other groups will look into these kinds of verbs more closely. 

But many verbs with this causative prefix have a different profile, like those in~(\nextx)--(\anextx). What can you say about their meaning (if anything)?
\pex
    \a \emph{gleiten} `to slide'
    \a \emph{begleiten} `to accompany'
\xe
\pex
    \a \emph{nehmen} `to take'
    \a \emph{benehmen} `to behave'
\xe

\section{Hebrew}

And is there a way to characterize the relationship between the simple verb and the causative verb in the following examples?

\pex \a \begingl
        \gla teo sagar et ha-delet//
        \glb Theo closed-\gsc{PAST} \gsc{ACC} the-door//
        \glft `Theo closed the door.'//
    \endgl
    \a \begingl
        \gla teo hesgir et axiv la-ʃiltonot//
        \glb Theo closed.\gsc{CAUS}-\gsc{PAST} \gsc{ACC} his.brother to.the-authorities//
        \glft `Theo turned his brother in.'//
    \endgl
\xe
\pex
    \a \begingl
        \gla teo matsa et ha-matana//
        \glb Theo found-\gsc{PAST} \gsc{ACC} the-present//
        \glft `Theo found the present.'//
    \endgl
    \a \begingl
        \gla teo hemtsi et ha-misxak ha-xadaʃ//
        \glb Theo found.\gsc{CAUS}-\gsc{PAST} \gsc{ACC} the-game the-new//
        \glft `Theo invented the new game.'//
    \endgl
\xe
\pex 
    \a \begingl
        \gla beki ravtsa al ha-ritspa//
        \glb Becky lazed-\gsc{PAST} on the-floor//
        \glft `Becky lay down on the floor', `Becky lazed around on the floor.'//
    \endgl
    \a \begingl
        \gla beki herbits-a la-xatul//
        \glb Becky lazed.\gsc{CAUS}-\gsc{PAST.3SG.F} to.the-cat//
        \glft `Becky hit the cat.'//
    \endgl
\xe


\section{Analysis}

What would a formal analysis look like? Specifically, what would the relationship be between the verb/root and the element \gsc{CAUS}?


\section{English}

If there's time, we can carry out a different kind of exercise.

Is \emph{kill} the same thing as \emph{cause to die}? Is \emph{melt} the same thing as \emph{cause to melt}? Let's look at some examples (some are mildly violent).

Direct causation:

\pex \a Mary caused John to die and it surprised me that he did so.
    \a \ljudge{*} Mary killed John and it surprised me that he did so.
\xe

Time adverbials:
\pex   \a Mary caused the glass to melt on Sunday by heating it on Saturday.
    \a \ljudge{*} Mary melted the glass on Sunday by heating it on Saturday.
\xe
\ex Pat broke the stick on Monday (*by stepping on it on last week).
\xe

Instrumentals and agent-oriented adverbials:
\pex \a The pirates caused Bill to die by swallowing his tongue.
	\a \ljudge{*} The pirates killed Bill by swallowing his tongue.
\xe

\ex Pat broke the stick quickly (*by stepping on it slowly). \xe

What do these examples mean for the relationship between the causing event and the caused event?



% \vfill

% If you finish with time to spare: what other kinds of causatives would you expect to exist?


\end{document}