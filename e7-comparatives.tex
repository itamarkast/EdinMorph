\documentclass[a4paper,12pt]{article}

%\usepackage{times}
%\usepackage[T3,T1]{fontenc}
%\usepackage[utf8x]{inputenc}
% \usepackage[T1]{fontenc}
% \usepackage{mathptmx}  

\usepackage{fontspec} %,xunicode} % xltxtra
    \defaultfontfeatures{Ligatures=TeX}
    \setmainfont{Brill}[
    Extension=.ttf,
    UprightFont=*-Roman,
    BoldFont=*-Bold,
    ItalicFont=*-Italic,
    BoldItalicFont=*-Bold-Italic,
    Renderer=ICU]
% 	\usefonttheme{serif}
%	\setmainfont[Renderer=ICU]{Charis SIL}
% 	\setmainfont[Renderer=ICU]{Brill}

%% Trigger answers and notes. Package by Byron Ahn with edits by Craig Sailor
\usepackage[]{myProbSol} %   Optional arguments:
%                                               'solution': reveals solutions
%                                               'spaces': leaves whitespace corresponding to the size of the solutions (has no effect if you also pass 'solution')


\usepackage{amsmath,amssymb} % for $\text{}$
\usepackage{url,natbib} 
\usepackage[dvipsnames]{xcolor} %,bm}
\usepackage{geometry,vmargin,setspace}
\usepackage{multirow}
\setmarginsrb{1in}{1in}{1in}{1in}{13.6pt}{0.1in}{0.1in}{0.2in}
% pdflscape} %rotating
    \setlength{\parindent}{0pt}

\usepackage{stmaryrd}
\usepackage{wasysym} %checkbox
\usepackage[normalem]{ulem} % for \sout{}
% \usepackage{mdwlist} % for \begin{itemize*} - but prefer \tightlist
\providecommand{\tightlist}{%
	\setlength{\itemsep}{0pt}\setlength{\parskip}{0pt}}

\usepackage{natbib}
	\bibpunct[:]{(}{)}{;}{a}{}{,}
	\setlength{\bibsep}{0pt plus 0.3ex}

\usepackage{longtable}
\usepackage[linkcolor=purple,citecolor=ForestGreen,colorlinks=true,urlcolor=gray,pagebackref=false]{hyperref}
\usepackage{multicol}
\usepackage{dashrule} %\hdashrule
\usepackage{array} % >{\...} in tabulars
\usepackage{arydshln}

\usepackage[framemethod=tikz,footnoteinside=false]{mdframed} 

\usepackage{expex} %Linguistics examples
\lingset{aboveexskip=0ex,belowexskip=0.5ex,aboveglftskip=-0.3em,interpartskip=0ex,labelwidth=!6pt,belowpreambleskip=0.1ex} %, *=*?}
%	\lingset{aboveexskip=0.5ex,belowexskip=0.5ex,*=??}

\usepackage[nocenter]{qtree} % trees
\usepackage{tikz}
    \tikzstyle{every picture}+=[remember picture]

\usepackage{tipa} % convenient for \textsubarch etc.

\newcommand\trace{\rule[-0.5ex]{0.5cm}{.4pt}}
\bibpunct[:]{(}{)}{;}{a}{}{,}
	\setlength{\bibsep}{0pt plus 0.3ex}

\usepackage{pifont}
\newcommand{\cmark}{\ding{51}}% 52
\newcommand{\xmark}{\ding{55}}%
 \newcommand{\hand}{\ding{43}}

\newcommand\zero{\O{}}
\newcommand\itp[1]{\textit{\textipa{#1}}}
\newcommand\gsc[1]{\textsc{\lowercase{#1}}} %for glossing in small caps - comment out to return to caps
\renewcommand\root[1]{$\sqrt{\text{#1}}$}

\newcommand\blue[1]{\textcolor{blue}{#1}}
\newcommand\red[1]{\textcolor{red}{#1}}
\newcommand\green[1]{\textcolor{ForestGreen}{#1}}
\newcommand\gray[1]{\textcolor{gray}{#1}}
\newcommand\denote[1]{$\llbracket$#1$\rrbracket$}
\newcommand\lra{$\leftrightarrow$}

\newcommand\vz{\text{Voice$_{\text{\{--D\}}}$}}
\newcommand\vd{\text{Voice$_{\text{\{+D\}}}$}}
\newcommand\pz{\text{$p_{\text{\zero}}$}}
\newcommand\va{\root{\gsc{ACTION}}}
\newcommand{\tkal}{\emph{XaYaZ}}
\newcommand{\tpie}{\emph{XiY̯eZ}}
\newcommand{\tpua}{\emph{XuY̯aZ}}
\newcommand{\thif}{\emph{heXYiZ}}
\newcommand{\thuf}{\emph{huXYaZ}}
\newcommand{\thit}{\emph{hitXaY̯eZ}}
\newcommand{\tnif}{\emph{niXYaZ}}
\newcommand\dgs[1]{\textsubarch{#1}}
\newcommand\del[1]{\sout{#1}}

\makeatletter %Allow superscript ^ and subscript _
\catcode`_=\active%
\gdef_#1{\ensuremath{{}\sb{#1}}}%
\catcode`^=\active%
\gdef^#1{\ensuremath{{}\sp{#1}}}%
\makeatother

\begin{document}
% \pagenumbering{gobble}
% \singlespacing


\hfill \emph{Morphology 2024-25, Edinburgh, Exercise 7}
\bigskip

The table below shows a number of adjectives in different languages: the regular ``positive'' form, the ``comparative'' form and the ``superlative'' form. For English, we see that while a lot of the affixation is regular, we sometimes encounter root suppletion. Let's notate the different stems (exponents of the root) with different letters: in the first two rows they're all A, but in the following two rows we have one A followed by two B's.
\bigskip

Complete the table.

\begin{center}
\begin{tabular}{lllll}
 & \textbf{Positive} & \textbf{Comparative} & \textbf{Superlative} &  \\\hline\hline
 English & \textbf{strong} & \textbf{strong}-er & \textbf{strong}-est & `strong'\\
 & A & A & A \\\hline
 English & \textbf{happy} & \textbf{happi}-er & \textbf{happi}-est & `happy'\\
 & A & A & A \\\hline
 
 English  & \textbf{far} & \textbf{farth}-er &  \textbf{farth}-st & `far' \\
 & A & B & B \\\hline
 
 French & \textbf{bon} & \textbf{mieux} & le \textbf{mieux} & `good'\\
 & A & B & B \\\hline
    % English  & \textbf{good} & \textbf{bett}-er &  \textbf{be}-st & `good' \\
 % & A & B & B \\\hline

 German & \textbf{schnell} & \textbf{schnell}-er & am \textbf{schnell}-sten & `fast'\\
    \\\hline


 French & \textbf{mauvais} & \textbf{pire} & le \textbf{pire} & `bad' \\
 \\\hline

  Latin & \textbf{bon}-us & \textbf{mel}-ior & \textbf{opt}-imus &  `good'\\
 \\\hline
% English  & \textbf{bad} &  \textbf{worse} & \textbf{wor}-st & `bad' \\

Danish  & \textbf{god} &  \textbf{bed}-re &  \textbf{bed}-st &  `good'\\
    \\\hline

German & \textbf{gut} & \textbf{bess}-er & am \textbf{bes}-ten & `good'\\
    \\\hline
    
Georgian & \textbf{k'argi}-i & u-\textbf{m\v{j}ob}-es-i & sa-u-\textbf{m\v{j}ob}-es-o & `good'\\
    \\\hline

Welsh  & \textbf{da} & \textbf{gwell} &  \textbf{gor}-au & `good' \\
    \\\hline

Basque & \textbf{asko} & \textbf{gehi}-ago & \textbf{gehi}-en & `a lot'\\
    \\\hline

Irish  & \textbf{maith} &  \textbf{ferr} & \textbf{dech} & `good' \\
    \\\hline
Persian  & \textbf{x\={o}b} &  \textbf{weh/wah}-\={i}y &  \textbf{pahl}-om/\textbf{p\={a}\v{s}}-om &  `good'\\
    \\\hline
Czech & \textbf{\v{s}patn}-\'{y} & \textbf{hor}-\v{s}í & nej-\textbf{hor}-\v{s}í & `bad'\\

\end{tabular}
\end{center}

What additional patterns emerge, beyond A-A-A and A-B-B? Which patterns might we expect to see, but do not find? Why not? Can you propose a structural explanation? You might find it convenient to assume that the root is spelled out as the positive form when it's a plain adjective, as well as two additional heads or features [\gsc{cmpr}] and [\gsc{sprl}].

% \bibliographystyle{linquiry2}
% \bibliography{lingxbib}

\end{document}