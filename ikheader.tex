\documentclass[xelatex,xcolor={dvipsnames}]{beamer}
\usepackage{times}
%\usepackage[utf8x]{inputenc}
% \usepackage[T1]{fontenc}
% \usepackage{mathptmx}  
 \usepackage{fontspec} % xltxtra
 \defaultfontfeatures{Ligatures=TeX}
%	\setmainfont[Renderer=ICU]{Charis SIL}
%	\setmainfont[Renderer=ICU]{Brill}
	\setmainfont[Renderer=ICU]{Libertinus Serif}
%	\setmainfont[Renderer=ICU]{Libertinus Sans}
	\usefonttheme{serif}
 %	\newfontfamily{\H}[Scale=1,Script=Hebrew]{David CLM} %{Frank Ruehl CLM}%{Ezra SIL}
\usepackage{colortbl,multirow,amsmath,amssymb,url,natbib,graphicx}
%\usepackage[dvipsnames]{xcolor}
%\usepackage{graphicx,colortbl,multirow,amsmath,amssymb,url} %natbib
\usepackage{pbox}
\usepackage{pifont}
%\usepackage{hyperref}
% \usepackage{geometry,vmargin,setspace,lscape} %rotating
\usepackage{xytree}
\usepackage{stmaryrd,wasysym}
\usepackage{arydshln}
\usepackage{array} %tabular column formatting
\usepackage{dashrule} %\hdashrule
\usepackage{siunitx} %decimals in tables

\bibpunct[:]{(}{)}{;}{a}{}{,} %\usepackage{dashrule}
		\setlength{\bibsep}{0pt plus 0.3ex}
\usepackage[normalem]{ulem}
\providecommand{\tightlist}{%
	\setlength{\itemsep}{0pt}\setlength{\parskip}{0pt}}

\usepackage{tipa}
\usepackage{expex}
	\lingset{aboveexskip=0.1ex,belowexskip=0.1ex,aboveglftskip=-0.1ex,interpartskip=0.1em,labelwidth=!6pt,belowpreambleskip=0.1ex} %, *=*?}
	\resetcountonoverlays{excnt}
\newcommand\trace{\rule[-0.5ex]{0.5cm}{.4pt}}
\usepackage[nocenter]{qtree}
\usepackage{tikz}
	\tikzstyle{every picture}+=[remember picture]

\newenvironment{dummy}{}{}
% \AtBeginSection[]{}
% \setbeameroption{show notes on second screen}
% \setbeameroption{show notes}

% \usetikzlibrary{shapes.geometric, arrows}
% \newlength{\bls}
% \setlength{\bls}{\baselineskip-\doublerulesep+1pt} % for double cline, uses package calc
% \newcommand\ipa[1]{\textipa{#1}}
\let\it\textit
\newcommand\zero{\O{}}
\newcommand\gsc[1]{\textsc{\lowercase{#1}}} %for glossing in small caps - comment out to return to caps
\newcommand\citey[1]{{\scriptsize #1}}
  \renewcommand\root[1]{$\sqrt{\text{\gsc{#1}}}$}
\newcommand{\cmark}{\ding{51}}% 52
\newcommand{\xmark}{\ding{55}}%
 \newcommand{\hand}{\ding{43}}

% \newcommand\hammer[1]{\begin{center}\fbox{\parbox{0.9\linewidth}{\centering \textsf{#1}}}\end{center}}
% \newcommand\hammersmall[1]{\begin{center}\fbox{\parbox{0.3\linewidth}{\centering \textsf{#1}}}\end{center}}
\newcommand\blue[1]{\textcolor{blue}{#1}} \newcommand\red[1]{\textcolor{red}{#1}}
% \newcommand\green[1]{\textcolor{ForestGreen}{#1}}
\newcommand\olive[1]{\textcolor{olive}{#1}}
 \newcommand\gray[1]{\textcolor{gray}{#1}}
\newcommand\denote[1]{$\llbracket$#1$\rrbracket$}

\newcommand\lra{$\leftrightarrow$}
\newcommand\dgs[1]{\textsubarch{#1}}
	\newcommand\vz{\text{Voice$_{\text{[--D]}}$}}
	\newcommand\vd{\text{Voice$_{\text{[+D]}}$}}
	\newcommand\pz{\text{$p_{\text{[--D]}}$}}
	\newcommand\va{\root{\gsc{ACTION}}}
	\newcommand{\tkal}{\emph{XaYaZ}}
	\newcommand{\tpie}{\emph{Xi\dgs{Y}eZ}}
	\newcommand{\tpua}{\emph{Xu\dgs{Y}aZ}}
	\newcommand{\mpua}{\emph{meXu\dgs{Y}aZ}}
	\newcommand{\thif}{\emph{heXYiZ}}
	\newcommand{\thuf}{\emph{huXYaZ}}
	\newcommand{\mhuf}{\emph{muXYaZ}}
	\newcommand{\thit}{\emph{hitXa\dgs{Y}eZ}}
	\newcommand{\tnif}{\emph{niXYaZ}}
\newcommand\del[1]{$<$#1$>$}

%=== shortcuts for ROOT  types ========== % 

%  explicit creation roots: <s, t> 
% (the s here is s_e), and the type of all our vPs
% we will have subscript versions of each type too
\newcommand{\set}{${{\langle}s_{e}, t{\rangle}}$} 
\newcommand{\setsub}{$_{{\langle}s_{e}, t{\rangle}}$}  % because I can't get these subscripted 

%  implicit creation roots: <e, t> 
\newcommand{\et}{${{\langle}e, t{\rangle}}$} 
\newcommand{\etsub}{$_{{{\langle}e, t{\rangle}}}$} % for subscripting

%  <e, <s_s, t>> for change of state roots 
\newcommand{\esst}{${\langle}e, {\langle} s_{s}, t{\rangle}{\rangle}$} 
\newcommand{\esstsub}{$_{{\langle}e, {\langle} s_{s}, t{\rangle}{\rangle}}$} 

\newcommand{\estsub}{$_{{\langle}e, {\langle} s, t{\rangle}{\rangle}}$} 

\newcolumntype{a}{>{\columncolor{red}}c}
\newcolumntype{b}{>{\columncolor{cyan}}c}
\newenvironment{rcases}
  {\left.\begin{aligned}}
  {\end{aligned}\right\rbrace}
%\tikzstyle{input} = [draw, inner sep=0.4cm, outer sep=0, minimum width=2cm, text height=1.6cm, align=center, text centered, draw=black, fill=blue!30]
%\tikzstyle{root} = [ellipse, minimum height=1cm, text width=1cm, align=center, text centered, draw=black, fill=green!30]
%\tikzstyle{lexeme} = [draw, inner sep=0.1cm, outer sep=0, text width=1.3cm, text height=0.2cm, align=center, text centered, draw=black, fill=orange!30, node distance=1.7cm]
%\tikzstyle{template} = [rectangle, rounded corners, minimum width=1.3cm, minimum height=1cm, text centered, draw=black, fill=red!30]
%\tikzstyle{label} = [font=\scshape]
%\tikzstyle{arrow} = [thick, ->, >=latex]
%\tikzstyle{relate} = [dotted, thick]

% \makeatletter %Allow superscript ^ and subscript _
% \catcode`_=\active%
% \gdef_#1{\ensuremath{{}\sb{#1}}}%
% \catcode`^=\active%
% \gdef^#1{\ensuremath{{}\sp{#1}}}%
% \makeatother

\setbeamercovered{transparent}
\usecolortheme[RGB={0,50,95}]{structure}
\setbeamercolor{alerted text}{fg=red}

\usepackage{etoolbox}
\makeatletter
\preto{\appendix}{%
	\patchcmd{\beamer@continueautobreak}{\refstepcounter{framenumber}}{}{}{}}
\makeatother

\setbeamertemplate{blocks}[rounded][shadow=true]
\setbeamertemplate{navigation symbols}{}
%\useoutertheme{infolines}
%\setbeamertemplate{frametitle}[default][center] 

%\usepackage{beamerthemeshadow}
\usetheme{Madrid}

\let\expexgla\gla % conflict with unicode-math package, which pandoc/beamer autoloads
\AtBeginDocument{\let\gla\expexgla}