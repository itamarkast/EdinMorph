\documentclass[a4paper,12pt]{article}

%\usepackage{times}
%\usepackage[T3,T1]{fontenc}
%\usepackage[utf8x]{inputenc}
% \usepackage[T1]{fontenc}
% \usepackage{mathptmx}  

\usepackage{fontspec} %,xunicode} % xltxtra
    \defaultfontfeatures{Ligatures=TeX}
    \setmainfont{Brill}[
    Extension=.ttf,
    UprightFont=*-Roman,
    BoldFont=*-Bold,
    ItalicFont=*-Italic,
    BoldItalicFont=*-Bold-Italic,
    Renderer=ICU]
% 	\usefonttheme{serif}
%	\setmainfont[Renderer=ICU]{Charis SIL}
% 	\setmainfont[Renderer=ICU]{Brill}

\usepackage{amsmath,amssymb} % for $\text{}$
\usepackage{url,natbib} 
\usepackage[dvipsnames]{xcolor} %,bm}
\usepackage{geometry,vmargin,setspace}
\usepackage{multirow}
\setmarginsrb{1in}{1in}{1in}{1in}{13.6pt}{0.1in}{0.1in}{0.2in}
% pdflscape} %rotating
\usepackage{stmaryrd}
\usepackage{wasysym} %checkbox
\usepackage[normalem]{ulem} % for \sout{}

\usepackage{natbib}
	\bibpunct[:]{(}{)}{;}{a}{}{,}
	\setlength{\bibsep}{0pt plus 0.3ex}

\usepackage{longtable}
\usepackage[linkcolor=purple,citecolor=ForestGreen,colorlinks=true,urlcolor=gray,pagebackref=true]{hyperref}
\usepackage{multicol}
\usepackage{dashrule} %\hdashrule
\usepackage{arydshln}

\usepackage{expex} %Linguistics examples
\lingset{aboveexskip=0ex,belowexskip=0.5ex,aboveglftskip=-0.3em,interpartskip=0ex,labelwidth=!6pt,belowpreambleskip=0.1ex} %, *=*?}
%	\lingset{aboveexskip=0.5ex,belowexskip=0.5ex,*=??}

\usepackage[nocenter]{qtree} % trees
\usepackage{tikz}
    \tikzstyle{every picture}+=[remember picture]

\newcommand\trace{\rule[-0.5ex]{0.5cm}{.4pt}}
\bibpunct[:]{(}{)}{;}{a}{}{,}
	\setlength{\bibsep}{0pt plus 0.3ex}

\usepackage{pifont}
\newcommand{\cmark}{\ding{51}}% 52
\newcommand{\xmark}{\ding{55}}%
 \newcommand{\hand}{\ding{43}}

\newcommand\zero{\O{}}
\newcommand\itp[1]{\textit{\textipa{#1}}}
\newcommand\gsc[1]{\textsc{\lowercase{#1}}} %for glossing in small caps - comment out to return to caps
\renewcommand\root[1]{$\sqrt{\text{#1}}$}

\newcommand\blue[1]{\textcolor{blue}{#1}}
\newcommand\red[1]{\textcolor{red}{#1}}
\newcommand\denote[1]{$\llbracket$#1$\rrbracket$}
\newcommand\lra{$\leftrightarrow$}

\newcommand\vz{\text{Voice$_{\text{\{--D\}}}$}}
\newcommand\vd{\text{Voice$_{\text{\{+D\}}}$}}
\newcommand\pz{\text{$p_{\text{\zero}}$}}
\newcommand\va{\root{\gsc{ACTION}}}
\newcommand{\tkal}{\emph{XaYaZ}}
\newcommand{\tpie}{\emph{XiY̯eZ}}
\newcommand{\tpua}{\emph{XuY̯aZ}}
\newcommand{\thif}{\emph{heXYiZ}}
\newcommand{\thuf}{\emph{huXYaZ}}
\newcommand{\thit}{\emph{hitXaY̯eZ}}
\newcommand{\tnif}{\emph{niXYaZ}}
\newcommand\dgs[1]{\textsubarch{#1}}
\newcommand\del[1]{\sout{#1}}

\makeatletter %Allow superscript ^ and subscript _
\catcode`_=\active%
\gdef_#1{\ensuremath{{}\sb{#1}}}%
\catcode`^=\active%
\gdef^#1{\ensuremath{{}\sp{#1}}}%
\makeatother

\begin{document}
% \pagenumbering{gobble}
% \singlespacing

\section{Affix ordering}

Japanese

Mirror Principle

Rice handout / \cite{rice11}: syntactic role (subj/obj), not semantic (e.g. passive agent), in Athapaskan.

Wakashan:
dress-make-past
sweets-taste-in.mouth-PERF-1PL.IMP
Yupik:
go-want-be.in.state-not-really-INTRANSITIVE.INDICATIVE-1SG‘I really don’t want to go.’


Bantu
(Paster paper?)


\subsection{Flipped orders (Rice's ab/ba)}

yupik
yug-pag-cuaryug-cuar-pagperson-big-littleperson-little-big‘little giant’‘big midget’

ayag-ciq-yugnarqe-ni-llru-u-qgo-future-probably-claim-PAST-INDIC.INTR-3SG‘He said he would probably go.’ayag-ciq-ni-llru-yugnarqe-u-qgo-future-claim-PAST-probably-INDIC.INTR-3SG‘He probably said that he would go.’

Oji-Cree
(14c)

Pulaar
(14d)

Bantu again
(14b)

Gombe Fula (22)

Japanese again
(from nyu/vienna stuff)


    \subsection{Phonological}
(18a): coronal before labial/velar
(18b): no non-coronal codas

But this is about infix and base. \cite{kalin20inf} on infixes as affixes.

(21a) sonority but irrelevant to semantic scope because never co-occur! (no data, check paper)
    Can mention Huave.
    maybe read Kevin Ryan for computational angle?

(24) Navajo, no scope

(26)-(27) Slave prosodic size of affix contra scope (look into this one)

(28) haplology

    \subsection{Templates}
Nimboran/Inkelas/Noyer

(51) has a nice quote from Hyman which is pro-templates

(56) mentions processing and social factors as possibilities.


\section{Allomorphy}
Just read \cite{gouskovabobaljik20cup}? \cite{bonetharbour12}? \cite{bobaljik00}?


\bibliographystyle{linquiry2}
\bibliography{lingxbib}

\end{document}
